%&14pt
\pdfhorigin=15mm \hsize=\pdfpagewidth \advance\hsize by-2\pdfhorigin
\pdfvorigin=15mm \vsize=\pdfpageheight \advance\vsize by-2\pdfvorigin
\parindent=0pt
{\bf Показать чему научился:}

$\diamond$ То что делали на первой неделе марафона.\par
$\diamond$ Туйшоу:
\parindent=28pt
\item{1.} ударять/отбивать в гунбу
\item{2.} четыре усилия в гунбу
\item{3.} горизонтальный круг в гунбу с переносом веса
\item{4.} фронтальный круг в гунбу в одну и другую сторону
(удар в лицо, отведение от живота; отведение от лица через шунь/ни, толчок в живот)
\item{5.} в мабу толкать в грудь---горизотнальный круг, вставить спичку (как на
тренировке в Томске---не в гунбу, чтобы научиться вращать только поясницей,
не трогая таза)
\item{6.} наматывать предплечьями со сжатыми кулаками как в первый день интенсива
\item{7.} свои запястья приклеены к запястьям партнера---круги двумя руками с шагами в сторону
\item{8.} уводим руки партнера вниз («вода») как во второй день интенсива
\parindent=0pt
\bigskip
\hrule
\bigskip
Три корня---ноги и крестец.
\bigskip
Туйшоу: горизонтальные круги одной рукой по 8 в каждой стойке (в обеих направлениях одноимёнными и
разноимёнными руками), потом с шагами вперёд/назад; вариант: вторую руку кладём на локоть
партнёру (если не достаём, то целимся наа локоть мысленно)
\bye
