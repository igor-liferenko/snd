%&12pt
\pdfpagewidth=297mm
\pdfpageheight=210mm
\pdfhorigin=1in
\pdfvorigin=0pt
\input QUIRE
\shhtotal=\pdfpagewidth
\htotal=.5\shhtotal
\vtotal=\pdfpageheight
\shoutline=0pt
\shstaplewidth=0pt
\shcrop=0pt
\shfootline={}
\shthickness=.27mm
\qtwopages \shipout\vbox{}

\horigin=10mm
\hsize=\htotal \advance\hsize by-2\horigin
\advance\hsize by-\QUIRE
\output={\ifodd\pageno\else\hoffset=\QUIRE\fi \plainoutput}

\vorigin=7mm
\vsize=\topskip \advance\vsize by37\baselineskip

\footline={\raise3pt\line{\hss\tenrm\folio\hss}}

\hyphenation{тай-цзи-цю-ань}

Итак, друзья, всем привет! Мы начинаем цикл лекций, посвящённых тайцзицюань. В дальнейшем 
предполагается на основе этого цикла лекций написать книгу о Тайцзи, но это уже будет чуть 
позже. Сегодняшняя лекция у нас вводная или обзорная и целью её является, скажем так,
предварительное знакомство с тем что такое тайцзицюань. То есть, я буду рассказывать с одной 
стороны исходя из того что вы можете совсем ничего не знать об этом явлении, с другой стороны 
попробую ответить на ряд вопросов, которые, возможно, являются не вполне очевидными даже для 
людей, уже практикующих и отчасти разбирающихся в тайцзицюань.

Прежде всего, название слова 
«тайцзицюань» --- это три иероглифа или можно сказать что это три слова: «тай» это великий, «дзи» 
это предел и «цюань» это кулак. Всё вместе стандартно переводится как «кулак великого предела», и 
соответственно можно сказать упрощённо что в своей основе тайцзицюань было школой 
кулачного искусства. По сути, школа кулачного искусства --- это то, что сейчас называется школой 
боевого искусства, поскольку китайцы это не различали. В китайском языке очень много 
пословиц, поговорок, связанных с идеей кулака. Кулак --- это аллегория бойца.
Классическая 
пословица, тоже кстати связанная с тайцзицюань: «Где ляжет бык --- там ударит кулак».
Имеется ввиду,
что человек, который практикует тайцзицюань, может тренироваться на пятачке, на котором 
укладывается корова. Поэтому «цюань» --- это не бокс в смысле использования рук в поединке. Это 
боевое искусство во всем его многообразии.

Соответственно, тайцзицюань изначально...\ ну можно 
сказать, что слово «тайцзицюань» --- это слово, которое не появилось с самого начала. Слово
«тайцзицюань» --- это уже достаточно позднее название. Более ранние названия были другими, 
использовались такие термины, как «кулачное искусство такого-то семейства». В частности,
считается исторически первой семьей, которая практиковала и развивала тайцзицюань, была 
семья Чэнь из местечка Чэнь Цзя Гоу, значит соответственно если я не путаю, провинция Хэнань.
Значит соответственно кулачное искусство семьи Чэнь. В некоторых источниках можно найти 
упоминание о «кулачном искусстве инь-янь». То есть были разные названия. Вот каноническое именно 
«кулачное искусство великого предела» --- это уже более позднее. Ну и отчасти можно сказать, что 
это такой маркетинговый продукт. Вот то есть, с одной стороны с самого начала тайцзицюань был 
заточен под определенную, скажем так, философию баланса. С другой стороны, максимально ярко 
эта философия баланса, то есть вот философия инь-янь, философия соразмерности, такое 
использование мягкого и жёсткого, использование круглого и квадратного, быстрые и медленные 
движения, и прочее прочее, это всё вот --- максимально ярко уже проявилась в XIX веке, когда
тайцзицюань стал приобретать не массовый характер, но тайцзицюань стал известным.

Массовый 
характер --- это уже XX век, это уже великие китайские социальные реформы, великие китайские 
потрясения, когда китайские коммунисты во главе с Мао Цзэдуном взяли курс на преобразование 
страны, и в частности было принято решение в середине XX века, что китайский народ нуждается в 
каком-то массовом методе оздоровления. Во многом идея массового оздоровления была взята из 
Советского Союза, то есть в каком-то смысле мы можем гордиться тем, что тайцзицюань, цигун, вот 
эти массовые явления, связанные со здоровьем народа --- это то, что появилось благодаря 
наблюдению китайцев за успехами Советского Союза в области оздоровления населения. 
Массовая физкультура, которая использовалась в Советском Союзе в 30-е, 40-е и далее годы: упор на 
какие-то массовые методы формирования представления о здоровье, использование радио для 
пропаганды каких-то принципов. То, что мы сейчас называем ЗОЖ, здоровый образ жизни, тогда 
это так не называлось. Вот это всё китайцы переняли из Советского Союза, ну и соответственно 
вот тогда уже тайцзицюань становится действительно массовым явлением. На сегодняшний день по 
разным оценкам тайцзицюань занимается от двухсот до трёхсот миллионов человек, то есть это 
соответственно самое массовое явление в истории человеческой цивилизации. Насколько я 
знаю, на втором месте, опять это очень примерные оценки, на втором месте вроде бы идёт айкидо, 
порядка 40 миллионов человек, на третьем месте идёт карате, порядка 20 миллионов человек. Хотя 
эти данные нужно ещё уточнять, перепроверять.

Я сразу предупреждаю: у меня как-бы нет цели 
дать какие-то сверхточные исторические справки. В любом случае, когда мы будем готовить книгу,
все факты будем проверять, перепроверять. Если сегодня в конце лекции кто-то там 
поправит меня... Можете писать, кстати. Кто не в курсе: можете писать свои вопросы в чате, который 
у нас прикреплён к каналу тайцзицюань. Ну, собственно, вот под тем объявлением, которое было 
о предстоящей лекции, можно писать свои вопросы или комментарии и, соответственно, я в конце 
лекции на возникшие вопросы отвечу. В том числе буду благодарен за какие-то правки и 
уточнения, если я где-то в чём-то ошибусь.

Ну вот, движемся дальше. Значит, когда зародился 
тайцзицюань. Есть разные версии эту тему. Я буду придерживаться более, скажем, научной версии.
Ещё буквально 30-40 лет назад, когда я сам начинал заниматься Тайцзи, очень сильная была 
дискуссия, в которой отстаивалась --- даже представителями исторической науки, в том числе 
Андреем Милюняком, одним из моих коллег и, скажем, старших товарищей по изучению тайцзи --- 
отстаивалась точка зрения, что тайцзицюань существует минимум шесть сотен лет, а то и 
больше. И соответственно вот эта красивая легенда про монаха Чжэнь Сэньфэна, который жил на 
горе Удан, соответственно изучал боевые искусства, соответственно монах Чжэнь Сэньфэн потом 
постепенно передал эту традицию боевых искусств вместе с даосской философией, вместе с 
представлениями даосской медицины другим людям, и так далее и так далее. Дело в том, что эта 
версия не подтверждается историческими данными. Что касается самого монаха Чжэнь Сэнфэна, 
да, возможно он существовал, но непонятно когда он существовал, по крайней мере на 
протяжении нескольких сотен лет. Встречаются упоминания разных монахов чжэньсэнфэнов, какой 
из них создал тайцзицюань, ну это большой вопрос, и создал ли вообще. Либо нам нужно признать, 
что какой-то из монахов жил более шести сотен лет, тоже непонятно. То есть это, скажем так, 
немножко за рамками официальной науки. Плюс у нас нет исторических подтверждений этому в 
документах.

Китайская культура очень необычная: китайцы привыкли 
фиксировать многое на бумаге и тщательно сохранять различные факты, различные ключевые 
события. Да, потом эти факты могли очень по-разному истолковываться. И в китайской 
историографии неоднократно известны случаи, скажем, подмены каких-то интерпретаций или 
замалчивания фактов, но есть --- вы удивитесь --- но есть такая наука история, и вместе с историей 
есть историография, источниковедение и так далее и так далее. И вот научный подход. В 
двадцатом веке этим занялись учёные и к концу двадцатого века выяснилось, что не было 
подтверждений, нет подтверждений из того, что нам известно из тех материалов. Нет
подтверждений тому, что был какой-то даосский монах шесть сотен лет назад и так далее.

Что было? Чему есть подтверждение? Значит, у нас есть подтверждение тому, что в деревне 
Чэнь Цзя Гоу был клан Чэнь. Чэнь Цзя Гоу буквально переводится как «овраг семьи Чэнь». Это 
историческое место. Мы там были. Изначально это была деревня, причём не очень крупная по 
китайским меркам. В деревне проживало может быть около тысячи человек. В 
конце 17 века по-моему 7 патриарх --- вот могу здесь соврать, но надо посмотреть будет поточнее 
потом --- седьмой патриарх клана Чень, значит Чэнь Ваньтин. Ему там стоит памятник в Чэнь
Цзя Гоу, он как раз практиковал боевые искусства, причём не просто практиковал боевые искусства.
Он был наставником по боевым искусствам китайской императорской гвардии. Соответственно, был то
что называется экспертом. Не просто мастер, а эксперт. Очень много знал разных школ, разных стилей.
Всю жизнь тренировался и выйдя в отставку, выйдя на пенсию, он решил заняться созданием своей
школы. Соответственно, здесь вот уже могут быть интерпретации позднейшие, но 
предположительно с одной стороны Чэнь Вантин ставил задачу создания эффективного боевого 
искусства, причём эффективного боевого искусства с уклоном в охрану, с уклоном в подготовку 
телохранителя-охранника.

Что такое работа телохранителя-охранника в позднесредневековом 
Китае? Это значит бой одного против толпы. В Китае того времени практиковались торговые 
караваны и торговые караваны нужно было охранять. То есть в Китае была довольно активная 
торговля. В отличие от Западной Европы, Китай был единым государством и соответственно 
торговые караваны могли путешествовать многие сотни километров. Соответственно их нужно 
было охранять от того, что хорошо описано в книге «Речные заводи», от лихих молодцов.
Причём лихие молодцы были иногда весьма искусно подготовлены в боевых искусствах, то есть
это не просто были бандиты, какая-то голытьба с палками в руках --- это были какие-нибудь
дезертиры из армии, это были какие-нибудь беглые монахи, ещё кто-то. В общем,
люди отчасти знакомые с 
боевыми искусствами. Вот против таких людей охранники караванов должны были защищать ценный 
груз и собственно торговцев. Ну, и караван не мог охраняться более чем несколькими десятками 
охранников, а нападающих могло быть несколько сотен. Поэтому тайцзицюань по своей тактике 
ведения боя никакого отношения не имеет к тому, что мы называем единоборством.

Если мы 
говорим о феномене единоборства, очень часто сравнивают тайцзицюань с современными 
спортивными единоборствами. Регулярно звучат идеи: ``Давайте сведём мастера Тайцзи с мастером 
MMA.'' Ну соответственно здесь сразу что нужно отметить: мы не знаем как будет работать 
Тайцзи против ММА в равных условиях. А что значит в равных условиях? Это когда у мастера 
Тайцзи во-первых в руках оружие, а без оружия никто там ничем не собирался заниматься, то 
есть это было просто бессмысленно. Все охранники были очень хорошо вооружены. И это раз. И 
два: мы не знаем как будет работать боец ММА в схватке один против толпы. То есть у нас просто 
нет такого примера ММА. Это классический бой один на один. Отдельная история, что там есть 
разные правила и так далее, но это можно опустить. То есть вот еще раз: тайцзицюань это не 
единоборство, это в основе своей как боевое искусство, это подготовка бойца для боя против 
толпы или для боя в толпе.

Я начинал только свое изучение тайцзицюань в девяностые 
годы прошлого века в Москве. Один из первых моих таких сильных учителей, кому я очень 
благодарен, был мастер ????? и значит параллельно с тем как он вёл у нас занятия по
тайцзицюань он например проводил семинары. Вот уже в 90-е годы проводил семинары в Москве для 
сотрудников служб охраны. Даже были разговоры, что он обучал какую-то крупную охранную 
структуру, то ли банковскую, то ли охрану. Каких-то первых лиц или что такое, и значит конкретно 
чему он обучал? Он обучал бою в лифте, бою в закрытых помещениях и бою одного против 
трех-четырех противников. Вот это такой контекст в каких условиях тайцзицюань максимально 
применим именно в современной реальности.

Возвращаемся к 
Чэнь Вантину. Конец 17 века. Есть с одной стороны идея, желание создать боевое искусство, которое 
будет практичным и будет применяться в подготовке вот этих вот охранников. Тогда это 
называлось охранные агентства, ну как по нашему можно сказать значит ЧОПы, частные 
охранные предприятия. Вот эти ЧОПы, они были по всей стране и крупнейшие охранные агентства 
насчитывали до нескольких тысяч бойцов. То есть это были не какие-то там маленькие структуры, а
весьма крупные объединения. Ну вот одно из таких охранных агентств как раз развивал 
Чэнь Вантин со своими учениками, ближайшими родственниками, сыновьями и последователями.
Соответственно на сегодняшний день к вопросу о преемственности поколений у 
нас есть прямая линия передачи традиции тайцзицюань буквально от отца к сыну. И если Чэнь 
Вантин у нас первый патриарх именно боевого искусства семьи Чэнь --- мы помним, что он был там 
седьмым патриархом вообще клана Чэнь, но вот в боевом искусстве он был первый --- соответственно 
на сегодняшний день у нас 20-е поколение --- мастер Чэнь Юй.

Ну дальше вот честно говоря я
не буду комментировать насколько это --- по крайней мере пока не буду комментировать --- насколько 
это применимо как охранное искусство в полном виде. Хотя ещё в середине 20 века тайцзицюань 
использовался в подготовке сотрудников охранных служб, использовался в подготовке отрядов 
полиции. И не только в Китае, но и в России, в Англии и в других странах. И не только тайцзицюань 
использовалось. Использовалось айкидо, использовалось багуаджан и так далее. То есть 
силовые структуры в общем не чужды боевым искусством. Особенно учитывая, что в китайских и 
японских боевых искусствах была разработана чёткая методика, что очень нравится 
представителям силовых структур. Четкость, организация, дисциплина и как-бы отсутствие 
анархии, отсутствие галимого творчества, то есть строгая последовательность упражнений, 
строгая последовательность методики, которая позволяет за понятные сроки подготовить из 
уже физически тренированного человека, подготовить бойца. По информации, которую я получил 
например от своего мастера Анатолия Григорьева можно сказать, что вот стандартом подготовки 
бойца для Чэнь Юя являлась --- не знаю как насчет сейчас, но раньше 20-30 лет назад мастер Чэнь Юй
говорил, что чтобы подготовить бойца ему нужно два года. Именно подготовить бойца, который 
будет в состоянии применить всё это как боевое искусство. Вот то есть вот это к вопросу о 
эффективности методики и каких-то стандартах подготовки бойца. Речь не идёт о десятке 
лет, как очень часто можно встретить упоминание в китайских текстах.

Китайцы обожают 
красивые такие метафоры и очень часто вы можете в китайских текстах встретить, например: 10 
тысяч раз какое-то движение нужно повторить. «10 тысяч раз» нужно понимать так, что иногда это 
действительно что-то осмысленное, то есть какое-то осмысленное число, иногда это 
буквально метафора «очень много». Слова «10 тысяч» в китайской культурной традиции являются ну 
примерным аналогом русского слова «тьма» --- тьма или очень много, несчетное число. Это с одной 
стороны. С другой стороны китайцы любят говорить например, что для того, чтобы стать 
мастером нужно 10 лет тренировок. 10 лет тренировок это значит очень долго, точно так же как 10 
тысяч часов значит очень много. 10 лет это значит очень долго. За 10 лет человек в китайской 
традиции меняется настолько, что он уже какой-то совсем другой человек. Поэтому, пожалуйста, 
не воспринимайте вот эти слова как буквально инструкцию или какой-то канон методики 
преподавания, что только через 10 лет можно говорить о мастерстве. Если человек регулярно 
тренируется, если человек регулярно применяет эти знания на практике, то речь идет не о 
десятке лет и уж тем более не о всей жизни. Речь идёт о двух-трёх годах. Это стандарт.

Опять 
возвращаемся к истории с Чэнь Вантином и, соответственно, созданию тайцзицюань. Опять-таки в 
современном таком культурном каноне бытует мнение --- и я долгое время это мнение разделял, 
поскольку у меня не было доступа к более, скажем так, надёжным источникам, каким-то научным 
исследованиям и прочее --- бытует мнение, что тайцзицюань изначально разрабатывался как школа 
с идеей дасской философии, с идеей какого-то глубокого баланса энергий, значит долголетия, 
продления жизни, и прочее прочее. То есть вот это сейчас не подтверждается. Сразу 
расставим точки над «ё», хотя я знаю, что многие занимающиеся Тайцзи сейчас на этом моменте 
начнут возмущаться: ``Нам учитель говорил по-другому, вы всё врете, это всё неправда.'' Ну вот, 
когда будем готовить книжку, наверное, попросим коллег, которые владеют источниковой базой, 
попросим, чтобы прям были ссылки на вот эти исследования китайских историков, потому что я 
не настолько хорошо владею китайским языком, чтобы это всё читать без перевода, поэтому есть 
конкретные исторические исследования, есть факты, которые свидетельствуют о следующем. 
Изначально была школа боевого искусства. Возможно, отчасти Чэнь Вантин был, скажем, близок 
даосскому мировоззрению. Вряд ли он был даосом --- у нас нет этому никаких подтверждений. То есть,
он не участвовал в каких-то религиозных службах, он не проходил обучение у каких-то даосских 
наставников, потому что даосская линия --- это прям ну тоже своя линия, там есть тоже система 
передачи.

И соответственно не пересекался даосизм на практике с Чэнь Вантином и с его 
последователями, но в какой-то момент те, кто развивали тайцзицюань --- возможно в 18 веке,
возможно в 19 веке --- возникла идея, что хорошо бы тайцзицюань соединить с даосской философией,
с даосским мировоззрением, с даосской медициной, опять-таки с учением о двенадцати меридианах, с 
учением о пустом и полном, и так далее. И соединить не просто на уровне идей, соединить 
практически. То есть в современном тайцзицюань буквально базовые упражнения даосской 
практики, даосской методики, саморегуляции используются именно для наработки техники 
боевого искусства. То есть на сегодняшний день это так, но как это было три сотни лет назад мы 
точно сказать не можем. Скорее не было чем было. Но это не умаляет того факта, что на 
сегодняшний день тайцзицюань представляет из себя в значительной степени синтез.

Здесь мы 
можем немножко отойти от истории и сказать, что современное состояние дел таково, что
тайцзицюань --- это триединый феномен. С одной стороны, это школа боевого искусства, причём
школа боевого искусства именно такого максимально практического, практически ориентированного, 
причём достаточно жёстко боевого искусства. Попытки превратить тайцзицюань в спортивное 
боевое искусство, в спортивное единоборство приводят к тому, что собственно тайцзицюань 
заканчивается. И там начинается какое-то не очень понятное перепихивание, переталкивание, 
что-то вроде то ли борьбы борцов сумо, то ли какого-то самбо или чего-то подобного. То есть 
изначально это было в основе своей боевое искусство, это первое. И было и остается сейчас. 

Второе --- это школа оздоровления. И в китайской традиции оздоровление в том числе предполагает 
продление жизни. Но опять-таки, вот опять давайте уточним, что по статистике у нас нет 
данных, что мастера тайцзицюань это какие-то супер долгожители. Да, совершенно точно были 
среди долгожителей Китая люди, которые занимались боевыми искусствами. Это факт. Но 
статистика, которая опубликована --- это сухие голые цифры. Можно с ними спорить, можно с ними 
не спорить, но это цифры. Соответственно статистика такова, что люди, занимающиеся боевыми 
искусствами, не живут дольше чем другие люди. Скорее долгожительство --- это результат 
определённой самодисциплины, определённого отношения к питанию, к режиму дня и так далее и 
так далее, вплоть до того, что это возможно жизнь где-нибудь в горных деревнях, где свежий 
воздух, где отсутствие стрессов, где человек в состоянии медитировать, как бы ничто человека 
не отвлекает, по большому счёту жизнь тихая, неспешная. Вот в таких условиях люди доживают до 
90 плюс лет. То что мы называем долгожительством --- свыше 90 лет. Но в этом смысле далеко не все 
китайские долгожители практикуют боевые искусства. То есть это факт. С другой стороны,
тайцзицюань всё-таки этому уделял внимание. Но вот в чём было это внимание --- ну например внимание
было в том, что необходимо чередовать медленные и быстрые движения. Быстрые движения 
используются для достижения результата для боя, для победы в поединке или в 
какой-то схватке --- для выживания. Медленные движения используются для восстановления 
после боя, для скорейшей реабилитации после каких-то травм.

А никакой человек, который 
занимается опасным охранным бизнесом не может обойтись без травм. Опять-таки красивые 
какие-то метафоры, красивые истории с жизнью мастеров Тайцзи, что они не позволяли коснуться 
себя. Ну что значит не позволять коснуться себя? Когда вы дерётесь против толпы, то может 
прилететь просто случайно. Да, вы непобедимы, но при этом просто случайно может прилететь, вы 
не заметите. Особенно учитывая, что у человека нет глаз вокруг, так сказать, головы --- и 
прилетит просто со спины, с затылка. То есть если где-нибудь в боксе, да и практически во всех 
современных, насколько я знаю, в спортивных единоборствах запрещены удары по затылку, то в 
реальном бою никто не будет ждать пока вы повернётесь лицом. То есть, как говорится, двое 
подходят спереди, третий, четвертый напрыгивают сзади и бьют по голове. Вот и всё боевое 
искусство. Поэтому травмы были. Соответственно, хороший мастер был не тот, кто не позволил 
себя ударить. Хороший мастер был тот, кто даже если пропустил какой-то удар, то пропустил его 
максимально безопасно для себя, без последствий, и потом быстро восстановился. Вот это очень 
важно. А для быстрого восстановления нужны были соответствующие методики. То есть, 
опять-таки, за счёт большого запаса здоровья, за счёт тренированности. Да, человек 
тренированный восстанавливается быстрее, чем другие. Но всё равно этого недостаточно. И 
характерно то, что в современных условиях Китая вот даже вот в этот последний всплеск 
коронавируса, китайцы на полном серьёзе использовали цигун и тайцзицюань для лечения 
и последующей реабилитации больных коронавирусом. То есть это до сих пор используется как 
официально рекомендованный одобренный метод помощи в медицинском лечении и реабилитации
после выздоровления, в период после болезни.

Итак, у нас получается второй компонент. Мы говорили, что триединая школа это: первое --- боевое
искусство; второе --- это глубокое 
оздоровление организма, продление жизни и быстрое восстановление после травм, реабилитация;
и третий компонент --- это духовное развитие, духовное самосовершенствование, развитие 
осознанности, развитие рефлексии, развитие того, что сейчас психология называет 
личностным ростом или личностным взрослением. Это опять-таки...\ вот я как профессиональный 
психолог могу сказать, что это не абстрактные какие-то слова или благие пожелания --- это 
конкретные методики, причём многие из методик личностного взросления, ну как минимум
насчитывают полторы-две тысячи лет. Когда я обучался на психфаке, я защищал как раз диплом по 
теме традиционное, значит развитие традиционного психологического знания в Китае
на рубеже с третьего века до нашей эры по третий век нашей эры, в этом периоде. И там вот вы 
можете не поверить --- да, ну я сам был в шоке, когда всё это узнал --- что две тысячи лет назад
китайцы уже знали многие вещи, которые сейчас психологи последние сто лет открывают: что есть 
характер у человека, что есть типы личности и что есть периоды формирования психики. И 
соответственно многое из того, что описано в современных учебниках по психологии, китайцы 
описали ещё 2000 лет назад. Уже начинали описывать соответственно идею вот 
этого духовного развития, духовного самосовершенствования. Она была встроена в школу тайцзицюань.

Опять-таки, нельзя сказать, что это было каким-то уникальным отличием школы тайцзицюань от других
школ боевых искусств. Это не было уникальным, это было характерным для 
многих школ, не для всех. Были школы боевого искусства, которые, например, не занимались, то 
есть не уделяли большого внимания вот этому оздоровлению и продлению жизни и так далее. Но 
были те, которые уделяли. Были те школы, которые уделяли внимание духовному 
совершенствованию, были те, которые никак к этому не относились. Школы были разные. На 
сегодняшний день мы знаем порядка шести сотен школ боевых искусств, о которых есть записи в 
исторических источниках. Шесть сотен школ боевых искусств. Понятно, что из этих сотен школ 
были те, которые вот по этому внутреннему устройству «боевое искусство, здоровье».
Соответственно духовный рост, духовное развитие ничем не отличались от тайцзицюань. Точно такие же
были. Да, была другая техника, но вот это деление на три компонента --- оно 
существовало и существует в других школах. Как минимум, мы можем привести примеры так 
называемых «внутренних школ», что само по себе тоже отдельный миф. Мы, наверное, в одной из 
лекций его коснёмся. Знаменитая группа школ внутреннего направления --- это тайцзицюань, 
синьицюань, багуаджан, но ещё иногда к этому относят люхе-бафа, но это уже кто как. Но вот как 
минимум три школы крупных --- все три школы крупные, все три школы известные и все три 
исторически никак не были связаны между собой. Синьицюань, тайцзицюань, багуаджан никак 
не были связаны изначально по отцам-основателям, по истории генеза. Да, в какой-то момент пошли 
пересечения, но не сразу. Тем не менее, во всех трёх школах у нас присутствуют вот эти три 
компонента. Поэтому это не какая-то уникальная особенность тайцзицюань.

Ну наверное для 
первого раза достаточно. Да, я как-бы наговорил определённый материал. Надо его переварить нам 
всем. Я уверен, что у вас есть вопросы. Я сейчас пойду посмотрю какие есть 
вопросы в чате. Если в чате нет вопросов, то соответственно можно будет 
задать свой вопрос вслух. Кто 
что-то хочет спросить или уточнить или прокомментировать. Вот есть поднятая рука. Антон,
пожалуйста, прошу. Включай микрофон, задавай вопрос. 

Андрей, вопрос такой. А какие виды оружия используются в тайцзицюань?

Ага, хорошо. Спасибо. По 
оружию у нас будет отдельная лекция, поэтому пока вкратце обозначу. Значит, клан Чэнь, то 
есть исходный тайцзицюань, практиковал, насколько я помню, 9 классических видов оружия. 
В основном это было оружие длинно-клинковое, то есть это длинномерное --- это было 
копье, это была алебарда, гуаньдао --- то есть это в нашем понимании что-то 
среднее между алебардой, глефой и так далее, соответственно была 
сабля дао, был меч цзянь, возможно были другие виды оружия. Но здесь у нас уже нет точной 
информации. Скорее всего был какой-то бой со щитом и мечом, возможно был какой-то бой с двумя 
саблями или с двумя мечами, но вот про это у нас нет точной информации, но у нас есть 
примеры комплексов. Например, на сегодняшний день известен комплекс таолу с парами дао, но с 
большой вероятностью это новодел. То есть, там есть ряд признаков чем отличаются современные 
комплексы-новоделы от старых классических форм, которые дошли до нас ну минимум там через
200--250 лет. 
Вот скорее всего парное оружие --- это уже новодел. Изначально копье, алебарда, сабля-меч, ну и 
возможно что-то ещё.

``Спасибо.''

``Так. Еще вопросы? Так, пожалуйста, Сергей.''

Андрей, спасибо за лекцию. В самом начале вы сказали, что возрождение тайцзицюань было 
направлено в первую очередь на оздоровление народа. Ну, был взят за пример Советский Союз, и 
физкультура...\ да-да-да, массовая, массовая. Наша массовая физическая культура 
транслировалась от нас к ним, и за основу была взята, ну, некая народная китайская система, 
скажем так, традиционная. С тех пор много прошло времени, и не является ли прикладной 
аспект, который, про который вы сейчас говорили, не является ли это тоже новоделом? Ну, я 
приведу пример, допустим, из айкидо. То есть, в айкидо, скорее всего, уисиба было тоже 
транслировано как некая физическая культура для оздоровления нации. Сейчас к нему пытаются 
прикрепить определённый, там, прикладной аспект, и появляется там реальное кедовращивочное,
ещё что-то. Вот не если сейчас, ну не если оздоровительные техники тайцзицюань переложены или 
прикручены в какой-то прикладной аспект вот этим же самым?

Ну вот смотрите... Спасибо, Сергей, за вопрос. То есть, наша школа...\ мы как-бы топим условно за
то, что называется как раз традиционный тайцзицюань, то есть мы не очень любим новоделы. Опять, я
не считаю, что плохо заниматься новоделами. Есть люди, которые всю жизнь посвятили каким-то 
современным версиям Тайцзи --- на западе называется модерн стайл, современный стиль ---
пусть занимаются, и наверняка от этого есть какая-то польза. И это, как говорится, всяко лучше, 
чем лежать на диване. Но есть каноническая школа, которая передавалась от поколения к 
поколению, от учителя к ученику. Вот это было с одной стороны ограничением китайских школ, ну 
не только китайских, японских, всех дальневосточных школ, а с другой стороны это спасло их от 
размывания и как бы от потери вот этого какого-то базового дискурса, базовых методик 
подготовки и так далее. То есть у нас есть фотографии значит конца --- ну когда у нас там 
фотография в Китае начала распространяться --- вторая половина или конец девятнадцатого века,
где мастера Тайцзи демонстрируют те же самые приёмы, которые демонстрируются сегодня. То 
есть, как минимум получается полтора века назад уже точно эти приёмы все были. У нас есть 
учебники по тайцзицюань, которые были написаны. Первый учебник по тайцзицюань 
вроде-бы был написан где-то в конце 18 века --- ну опять, по тем данным которые у нас есть --- и
там уже 
как раз упоминаются приёмы, конкретные приёмы, конкретные техники именно боевого искусства. 
Они ничем не отличаются от современных. Ну да, понятно, что там можно было пытаться переврать 
это как-то и так далее но названия все те же самые, были картинки, что-то сохранилось в 
картинках, что-то нет, что-то сохранилось. То есть вот это всё было реконструировано и, что 
важно, была прямая передача, то есть если вот например в отношении шаолинь прямая передача 
была нарушена, при всём при том, что мастера шаолинь на сегодняшний день действительно есть,
но прямая передача в шаолинь была нарушена --- это факт исторический, с этим 
бесполезно спорить, просто множество источников это показывают --- то в тайцзицюань прямая 
передача не была нарушена и она длится вот уже 20-е поколение мастеров Тайцзи. Это чётко от отца к 
сыну всё передавалось. То есть даже если была какая-то в XX веке тема новодельная, кто-то что-то 
придумывал, фантазировал, как например товарищ Кадочников что-то там воссоздал, скорее всего 
в том числе на базе Тайцзи. Не то что это плохо, но определённо это новодел. Но в тайцзицюане 
прямая линия прослеживается. Вот мы как раз вот этим и занимаемся --- тем, что отчасти пытаемся 
реконструировать, отчасти просто аккуратно воспроизводим те методики, которые до нас дошли. 
Ответил я наш вопрос?

``Да, спасибо.''

Наверное последний вопрос, если у кого-то есть, и будем 
сегодня завершать. С шутками и прибаутками уже целый час беседуем. Ну, видимо, тогда всё. Всем 
спасибо. Пишите вопросы по тому материалу, который уже я наговорил. Пишите вопросы, которые 
вас интересуют по тайцзицюань. То есть, с одной стороны, мне есть о чём говорить, и в черне уже 
план лекций подготовлен. С другой стороны, ваши вопросы, безусловно, помогут сделать и 
лекции, и будущую книжку более интересной, более полезной, практичной и содержательной. 
Планируются лекции раз в неделю, по субботам, примерно в то же самое время, но могут быть 
небольшие подвижки. Мы будем оперативно вас оповещать. На сегодня всё. Всем счастливо!
\bye
