%&12pt
\pdfpagewidth=297mm
\pdfpageheight=210mm
\pdfhorigin=1in
\pdfvorigin=0pt
\input QUIRE
\shhtotal=\pdfpagewidth
\htotal=.5\shhtotal
\vtotal=\pdfpageheight
\shoutline=0pt
\shstaplewidth=0pt
\shcrop=0pt
\shfootline={}
\shthickness=.27mm
\quire{4}

\horigin=9mm
\hsize=\htotal \advance\hsize by-2\horigin
\advance\hsize by-\QUIRE
\output={\ifodd\pageno\else\hoffset=\QUIRE\fi \plainoutput}

\vorigin=3.8mm
\vsize=\topskip \advance\vsize by37\baselineskip

\footline={\raise1pt\line{\hss\tenrm\folio\hss}}

\font\titleF=omssdc10 at14pt

\noindent{\titleF Глава 1. Знакомство с миром Цигун и Дао}

АБ: Давай начнем с определения Ци.
Что это такое, как мы узнаем о ее существовании,
как она проявляется? Как ее использовали даосы?

АШ: Давай просто скажем, что такое Ци с точки зрения
современного научного понимания. Как там
раньше считали даосы --- это отдельная тема, но
современное понимание Ци такое.
В теле постоянно что-то меняется: какие-то
процессы идут, что-то нагревается, что-то
остывает, давление где-то повышается, где-то
снижается. Где-то, говоря медицинским языком,
функциональность повышается или снижается.
Если почки начинают активно качать жидкость, ---
то мы это чувствуем и говорим: «о, почки
заработали», --- а китайский специалист скажет:
«меридиан почек активизировался». В рамках
концепции «Ци» --- это просто перемены в теле.
И эти перемены бывают как благотворные, так и
деструктивные. Деструктивные --- это когда где-то
что-то застревает. Функциональность, мета болизм, циркуляция жидкостей, эмоциональные
или нейрохимические процессы.

АБ: При этом я правильно понимаю, что Ци --- это
динамика?

АШ: Да, это динамика.

АБ: Динамика может быть плохая, когда идет что-то
неправильно, может отсутствовать динамика, и
это тоже плохо. Иногда может быть хорошо, а
иногда плохо?

АШ: Мы говорим, что это застой какой-то, а застой
ведет к деградации, снижению
функциональности на физическом уровне, и
постепенно мозг и все остальное начинает
страдать.

АБ: Тогда, значит, получается так: мы вполне
способны, с опорой на современную медицину,
сказать, какая должна быть динамика с точки
зрения биохимии организма. С какой частотой
должны включаться почки и все остальное. Такая
примерно идея?

АШ: Я не настолько хорошо разбираюсь в медицине,
не получал систематического медицинского
образования. Анатомия ЦНС, Физиология ВНД,
курсы по патофизиологии, семестр в
медучилище, самообразование.
Я не возьмусь однозначно утверждать, тут надо
привлекать других экспертов. У меня есть
знакомые, которые одновременно владеют цигун
и имеют высшее медицинское, ведут практику.
Можно будет их привлечь позже.

Сейчас пока что важно для нас? Важно, что у нас
в теле эти перемены идут сами собой. Организм ---
сбалансированная экосистема. Идет метаболизм,
идут процессы жизнедеятельности. И если не
мешать, то теоретически, все само собой
наладится. Например, если человек вовремя и
качественно питается. Понятно, что, если он
голодает, --- организм начнет буквально пожирать
сам себя. И в итоге --- это нехорошо для здоровья.
И наоборот, если человек переедает --- это вредно,
будет застой.
«Цигун», дословно, это «работа ци» или «работа
с ци».

Наша задача, когда мы работаем с ци... в идеале, ---
просто не мешать организму, чтобы он поспал,
поел, погулял, получил новые впечатления,
чтобы в кровь выбрасывались соответствующие
гормоны, прежде всего --- гормоны
«положительные». А, соответственно, гормоны
стресса чтобы вырабатывались поменьше. То есть
они тоже должны вырабатываться, иначе без них
и радости не будет. Но при этом человек не
должен жить в постоянном стрессе.
И здесь нам поможет метафора про «прежнее
Небо» и «последующее Небо».

Вся Ци делится на 2 категории. Ци прежнего Неба
и Ци последующего Неба.
Когда мы в стрессе, мы тратим
неприкосновенный запас ци. Тоже самое, когда
мы голодаем, когда недосыпаем систематически,
когда мы мало гуляем, организм начинает
тратить неприкосновенный запас ци, который
дан нам от рождения. Эта особая ци называется
Юань Ци (Ци прежнего Неба), и она является
семенем жизни. Это такая маленькая супер батарейка, на которой мы всю жизнь живем.

АБ: В одной из наших бесед мы с тобой говорили о
том, что есть несколько этапов движения по
традиции.

\item{$\triangleright$} Лечение психики, лечение головы; человек при
этом, за счет внутренних процессов перестает себя
заморачивать и мучать.

\item{$\triangleright$} Создать организму условия жить нормально, как
минимум. Достижение оптимального здоровья
для своего возраста. То, что напортачил к этому
возрасту, как-то вернуть, оздоровиться.

\item{$\triangleright$} И только после этого наступает этап накопления
излишков энергии, которую на 4 этапе можно уже
переводить в Юань Ци и поворачивать вспять
старение. Такова примерно общая схема?

АШ: Я буду откровенен.
Первый этап --- это вербовка. Прежде, чем с тобой
начнет происходить что-то хорошее, ты должен
быть завербован.

АБ: Вербовка куда?

АШ: А вот это интересно.
Первая идея, «первое правило Даосского Клуба»:
ты должен признать, что ты в тюрьме. Это уже
определенная и весьма сильная картина мира.
Второй момент: ты должен признать, что жизнь в
тюрьме можно сделать комфортной (прекратить
страдать и болеть), и, в дальнейшем, из тюрьмы
можно «свалить». Причём, свалить из тюрьмы, не
умирая физически. Свалить именно в этом теле. В
даосских трактатах это прямо называется:
«воровской побег на небо»...

Пока этой картины мира в голове не будет --- всё
бессмысленно. Потому что про практики Цигун я
бы сказал так: таблетки --- не хуже помогают. По
крайней мере, в ближайшие 5--10 лет, здесь и
сейчас. Вот ты, к примеру, таблетки принимаешь,
а кто сказал, что от них вред? Потом когда-то от
них будет вред, а может и не будет... Может,
другие таблетки разработают. Ты понимаешь, о
чем я? Это вопрос веры, вопрос картины мира...

{\bf Три пути для даосского «воровского побега на Небо»}

1. Внутренний Путь. Цигун, боевые искусства,
хитрые позы, упражнения.

2. Внешний Путь: ты должен что-то съесть, но
приготовить должен сам, т.к. трудно доверять
другому (пилюля бессмертия --- ядовитая, можно
умереть). Парашют должен складывать сам, без
претензий, если не раскрылся. В народе это
называется Путь Вина. Даосы часто рисуются в
народных преданиях с бутылкой. Вино помогает
сохранять состояние легкого опьянения, ища путь
в небо. Так или иначе, на этом пути ты
употребляешь, принимаешь внутрь какие-то
вещества.

3. Путь парной практики, сексуальной практики
с другим полом. Третий путь --- самый быстрый.
Если 1 и 2 пути занимают минимум по 10--20 лет,
то этот путь занимает несколько лет. И
одновременно - самый тяжелый, т.к. нужна
идеальная партнерша, подходящая по многим
параметрам. Партнёры должны быть духовно
близки... и при этом не должно быть взаимной
привязанности. Но это --- почти невозможно.
Обычно люди либо не близки друг другу и легко
расходятся, либо близки и привязываются. А с
привязанностью ничего не получится. Даосы
многократно создавали семьи, но долго там не
задерживались. Тем более, что Даос живет
больше 100 лет, поэтому можно побыть женатым
10--15 лет...

{\bf Из чего состоит даооское саморазвитие?}

Есть три составляющих процесса
саморазвития --- это Правильное Питание,
Правильное Поведение и Правильное Мышление.
Правильное Питание --- это все, что ты в себя
«кладешь»: еда, вода, вино, таблетки, травы,
лекарства --- это все относится к «питанию». Это
восполняет твою жизненную энергию, можно
сказать, на «плотном» уровне. Твое «плотное»
тело получает при этом правильный баланс
энергии.

Правильное Поведение --- это тренировки, режим
дня, практики. Это позы, которые ты в течение
дня принимаешь, это осанка, и походка. Я бы
сказал, это много-много микро-стратегий, много-много
привычек, которые встраиваются в повседневное поведение.
Как про НЛПеров говорится: у НЛПера все
пальцы в якорях, весь обвешенный якорями...
Даосы --- они тоже такие: все обвешенные микростратегиями правильного поведения. Причем,
очень подробно, в деталях все это делается. Мы в
своё время с Ю.А. Чекчуриным спорили на эту
тему. Он говорил: «У даосов, у твоих, обобщение
преобладает!» А я смотрю, в реальной практике
наоборот: детализация, детализация и ещё раз
детализация! Там обобщение --- ну, это так, для
мирян. Сказал «Дао» --- не сказал ничего. Для
мирян --- обобщения: мир несовершенен!

А как только ты начинаешь практиковать, тебе
говорят: «Неправильно! Палец сюда!» А почему
сюда? А потому, что если сюда, то будет вот так! И
так каждый палец на место ставится, каждый
сустав, каждая часть тела.

И третий компонент --- это Правильное
Мышление. В вопросах правильного мышления
даосы практически согласны с буддистами, за
исключением того момента, что буддисты
говорят: «тело --- не важно». А даосы говорят:
«Нет, тело важно!» Поэтому, правильное
мышление --- и про тело тоже. Нужно следить за
телом, при этом быть непривязанным, быть
пустотным, быть все время в потоке перемен. Ни
на чем мысленно не застревать, и следовать своей
безумно далекой цели.

Если говорить про правильное мышление по
психотипам... я бы сказал, что даосский психотип
--- это параноял в сочетании с эпилептоидом.
Параноял потому, что есть безумно далекая цель,
практически для тебя не достижимая, но нужно к
ней идти. А эпилептоид --- потому, что много-много
мелких действий, которые нужно ежедневно повторять. Это к вопросу: какой
психотип формируется у человека, если он идет
по этому пути.

\bye
