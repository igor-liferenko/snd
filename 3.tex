%&12pt
\pdfpagewidth=297mm
\pdfpageheight=210mm
\pdfhorigin=1in
\pdfvorigin=0pt
\input QUIRE
\shhtotal=\pdfpagewidth
\htotal=.5\shhtotal
\vtotal=\pdfpageheight
\shoutline=0pt
\shstaplewidth=0pt
\shcrop=0pt
\shfootline={}
\shthickness=.27mm
\qtwopages \shipout\vbox{}

\horigin=10mm
\hsize=\htotal \advance\hsize by-2\horigin
\advance\hsize by-\QUIRE
\output={\ifodd\pageno\else\hoffset=\QUIRE\fi \plainoutput}

\vorigin=7mm
\vsize=\topskip \advance\vsize by37\baselineskip

\footline={\raise3pt\line{\hss\tenrm\folio\hss}}

\hyphenation{тай-цзи-цю-ань}

Итак, друзья, всем добрый вечер. У нас сегодня третья лекция. Мы сегодня будем говорить о том, 
что такое центр тела и движок Тайцзи. Ну, собственно, мы уже начали говорить о центре тела на 
прошлом занятии, на прошлой лекции, где говорили о проблеме преподавания тайцзицюань, 
проблеме перевода тайцзицюань. Говорили о том, что такой ключевой момент тайцзицюань, как 
работа поясницы или роль поясницы, с большим трудом, с проблемами переводится. В принципе. И 
если делать какой-то смысловой перевод, с этим есть проблемы. Но сегодня поговорим уже, 
скажем так, не в том ключе, какие проблемы у переводчиков, какие проблемы у преподавателей.
Поговорим о том, а что, собственно, мы изучаем с точки зрения именно тела, когда изучаем 
Тайцзи.

Сразу не помню, говорил я или нет, когда мы говорим Тайцзи, это буквально 
«великий предел», это сокращение. Слово «тайцзицюань», во-первых, длинное, во-вторых, мне каждый 
раз рот ломать не хочется. Поэтому я сокращаю до Тайцзи или Тайчи. Это абсолютные аналоги, 
хотя встречаются очень странные заявления, что Тайчи --- это что-то совсем другое. Ну и можно, 
конечно, говорить тайцзицюань, но это просто долго. Поэтому в данном контексте Тайцзи и 
тайцзицюань --- это одно и то же.

Соответственно, как у нас работает тело в тайцзицюань? Мы 
исходим из того, в тайцзицюань, что у нас есть определённая структура тела, которая сначала 
достаточно длительное время нарабатывается, и потом, когда эта структура наработана, вы в 
этой структуре двигаетесь, уже не затрачивая специальных усилий, не отвлекая своего 
внимания на то, чтобы эту структуру поддерживать. Какая это структура? Ну, прежде всего, мы 
должны сказать, что это в статике столбовая структура или джан-джуан, столбовое стояние, 
столбовая практика. Столбовая структура --- это, с одной стороны, не прерогатива тайцзицюань, 
то есть столбом стоят во многих школах ушу, и столбовая практика --- по некоторым данным, 
опять-таки, есть дискуссии на эту тему, но специалисты считают, что столбовая практика --- это 
более древний метод тренировки, нежели Тайцзи. То есть, если мы говорили, что Тайцзи по 
официальным научным данным, которые имеют какое-то подтверждение фактологическое, это 300 
лет, то, соответственно, столбовые практики в разных школах описаны 700--1000 лет, возможно, 
полторы тысячи лет, в зависимости от того, что считать «столбом».

На сегодняшний день мы 
говорим о джан-джуан, столбовом стоянии, как о позиции тела, в которой тело, с одной стороны 
устойчиво, с другой стороны может быть в движении. То есть это тело не закрепощённое в 
столбе, устойчивое, имеет хорошую опору, и при этом большая часть мышц расслаблена, работают 
только те мышцы, которые здесь и сейчас поддерживают равновесие. Если вы практикуете столб 
хотя бы несколько месяцев, то вы спокойно можете в этом положении сидеть от 15 минут до часа. 
По большому счету, если есть базовая наработка столба, то есть если вы в принципе в столбе 
сидите, если вас туда посадили, если вам объяснили, поправили какие-то элементарные вещи и так
далее, то дальше только вопрос терпения и силы воли. То есть не требуется какая-то физическая 
сила, не требуется какая-то суперкоординация и так далее. Базовая столбовая стойка, 
столбовая работа --- это, ну, если вы можете сидеть 15 минут, то есть если у вас настолько 
здоровья хватает, то вы можете сидеть и час тоже. Только исключительно силой воли и 
терпением. А если вы можете сидеть час, то, в принципе, дальше уже неважно, сколько вы сидите. 
То есть даже в старое время, в тех школах, в которых я занимался, говорили, что раньше сидели 
столбом два часа, значит, для того, чтобы что-то там выстроить, наработать. А в современных 
школах традиционных, ну, современных версиях традиционных школ, можно так сказать, 
считается, что 40 минут достаточно. То есть если вы сидите 40 минут, то у вас уже большинство 
ошибок выправляются, значит, и дальше уже можно практиковать самостоятельно.

Опять-таки 
есть ошибки базовые, которые, ну, скажем, идут от невнимательности. Например, давайте 
пройдёмся по ключевым требованиям столба. Значит, начнём снизу, пойдём вверх. Стопы. Стопы у 
нас должны быть, с одной стороны, расслабленные, то есть не поджатая стопа, не стиснутые 
пальцы и так далее. Китайцы говорят, опора на 9 точек стопы. Это значит, на каждой стопе у вас
упираются в пол все пять пальцев, плюс в пол опираются, давят, точки в основании пальцев. Это, 
можно сказать, самая широкая часть стопы. Проекция направо-налево по стопе точек юнцюань. То 
есть вот это две точки. Плюс ещё пятка справа, пятка слева. Ну или внутренняя часть пятки, 
наружная часть пятки. Получается 9 точек на каждой стопе. В реальности не требуется, чтобы 
все эти 9 точек прям давили в землю, давили в пол. Достаточно, если вы будете чувствовать 
опору пальцами и если вы будете чувствовать опору внутренней частью пятки. Скорее всего, при 
этом все остальные части стопы тоже будут на месте.
Соответственно, дальше мы идём.
Вот у нас стопа укоренилась, пустила корни, мы ощущаем, вот здесь тоже такая полезная 
метафора или полезная фраза: «мы стоим не на ногах, мы стоим на полу». То есть когда вы стоите в 
столбе, когда вы делаете Тайцзи, то вы стоите не на ногах. То есть ваша задача не ноги ощущать. 
Ваша задача ощущать пол, на который опираются ноги. В этом есть разница.

Соответственно, 
поднимаемся вверх. У нас колени. Очень важный момент --- это то, что колени выстраиваются без 
усилия. То есть если стопу, например, мы давим в землю, то есть это определённое усилие в 
стопе, то колени --- мы скорее следим за тем, чтобы колени были в правильном положении. 
Правильное положение колена --- это когда колено смотрит примерно туда же, куда носок стопы. С 
одной стороны. С другой стороны, колено не заваливается слишком сильно вперёд и не выпрямляется 
слишком сильно назад. Понятно, когда у вас нога выпрямлена, то колено как бы подано назад. 
Когда нога слишком сильно согнута, так что вы не видите за коленом носок стопы, то это тоже 
ошибка. То есть вот колено должно быть сбалансировано. И мы его никуда не давим. То есть 
колено --- это такой уязвимый сустав. Если его начать давить, то оно будет закрепощаться в 
неестественном положении. И дальше при каких-то рывках, усилиях и так далее колено будет 
травмироваться. Поэтому колено должно свисать. Как говорят китайцы: «колени смотрят вниз». 
Но это бывает непонятно --- что такое «колени смотрят вниз». Они же вперёд смотрят. Поэтому это 
скорее вопрос ощущений, что вы настраиваетесь на то, чтобы колено расслабилось и опустилось 
вниз, обмякло.

Дальше у нас таз. Таз у нас стянут сзади или поджат сзади, сгруппирован, собран. 
То есть ягодицы должны быть в тонусе. Не стесняйтесь, в столбе потыкайте себя буквально 
сзади пальцами в ягодицы --- убедитесь, что справа-слева две большие ягодичные мышцы в тонусе,
что они такие не вялые, не проваливаются. При этом копчик --- это то, что между ягодицами, та
часть 
позвоночника, которая подворачивается, как-бы уходит между ягодицами к промежности ---
копчик должен быть свободный.

Спасибо, что задаёте вопросы. Я обязательно на все вопросы 
отвечу. Если я какие-то вопросы забуду ответить, прокомментировать, в конце ещё раз, 
пожалуйста, напомните.

Сейчас пройдём последовательно по всем требованиям столбовой 
стойки. Можно как раз на часть вопросов ответить. Значит, итак, ягодицы поджатые, копчик 
свободный. Соответственно, здесь уместно говорить о том, что можно объяснить как правильно, 
но некоторым людям понятнее, если объяснить как не правильно, то есть как не надо делать. 
Значит, две основных ошибки --- это откляченный копчик или оттопыренный, такой, таз назад. И 
вторая ошибка --- поджатый копчик, то есть поджатые не только ягодицы, а именно сам копчик, как 
будто мы поджали хвост. Здесь нужно опустить копчик вниз, но опять, он не может совсем 
опуститься вниз. Когда в китайских учебниках пишут, что нужно опереться на копчик, то это 
немножко странно. Копчик у нас всё время подвёрнут. Скорее, область копчика должна быть 
расслабленная. Здесь опять вопрос терминологии. И когда мы говорим, что мы, например, 
опираемся на хвост, то правильнее говорить, что мы опираемся на хвост в районе крестца. 
Можете себе представить, где у вас там крестец, посмотреть по медицинским каким-то 
картинкам. Так вот, крестец мы представляем себе, что он как третья нога, как у бобра или как у 
кенгуру, опирается в землю. Копчик подвёрнут, но область копчика свободная. Значит, вот мы 
выстроили таз. Обратите внимание, если вы правильно выстроили таз, то есть сгруппировали таз 
сзади и спереди просто расслабили. Важно, что спереди таз не зажатый, не закрытый. Можно 
даже пальцами себе потыкать в область бёдер в верхней части и в низ живота. Там должно быть 
мягко, то есть не жёстко. Если вы это сделали, то у вас естественным образом ноги поджимаются 
сзади и расправляются спереди. Естественным образом разворачиваются колени. Вот опять 
принципиальная ошибка у людей, которые смотрят на конечный результат: ``Ага, у мастера колени 
наружу, значит я тоже буду давить колени наружу.'' При том, что мастер никаких давлений на 
колени не окажет. Ученики начинают мастера как-бы слепо копировать, не чувствуя и не понимая,
как там это 
всё выстроено.

Поэтому колени свободные, таз правильно выстроенный, и мы тазом уверенно 
опираемся вниз на пол. Ну, ещё иногда говорят, что правильная позиция таза в столбе --- как-будто 
вы сидите на барной табуретке. Подчёркиваю, табуретке, то есть у спины нет опоры. Значит, вот, 
соответственно, вы сидите на барной табуретке, она высокая... Я, например, не хожу по 
барам, и поэтому мне этот образ не понятен, я своим ученикам говорю другой образ: присядьте на 
подоконник или на батарею. Ну, в зависимости от вашего роста. Если вы ростом, как я, метр 
восемьдесят, то вы можете присесть на подоконник. Вот это ощущение, что вы опираетесь тазом 
на подоконник, и при этом вам удобно сидеть, у вас нигде ничего не напрягается. Если вы ростом 
метр шестьдесят, метр семьдесят, то присядьте на батарею, вот то же самое ощущение, вот это 
ощущение опоры тазом чуть-чуть вниз и назад.

Значит, дальше у нас идёт поясница-живот. Поясница у нас плотная, и поясничный изгиб --- у нас
есть естественный поясничный изгиб --- вот нам его нужно минимизировать. То есть, с одной стороны,
в столбе позвоночник, скажем так, в естественном положении, то есть он не перенапряжённый, с
другой стороны, столбовое настаивание предполагает, что вы поясницу чуть-чуть выдавливаете
назад, то есть в ней возникает такое упругое наполнение. Китайцы говорят: как-будто вы опираетесь
на
подушку, набитую конским волосом. Ну, опять-таки, может быть не понятно, что такое «подушка,
набитая конским волосом». Ну, вот достаточно, если вы, например, в машине, значит, сядете в
кресло, и при этом сможете надавить поясницей на кресло, то есть у вас не будет
провала между поясницей и креслом. Второй
вариант, это когда вы опираетесь...\ то есть, выставляете ноги немножко вперёд, опираясь на стенку
или на шкаф какой-нибудь, на дверь. Ноги отставляете от стены и так же поясница плотно прижата к
стене, к опоре, то есть нет зазора между поясницей и стенкой. Вот это ощущение наполненной
поясницы. В столбе, поскольку мы не можем надавить ногами назад, как это делаем рядом со
стенкой, в столбе мы чуть-чуть втягиваем пресс --- мышцы живота --- таким образом, чтобы мышцы
живота толкнули поясницу назад. То есть, у нас уменьшается вот этот поясничный прогиб, не полностью
уходит, но уменьшается. Вот. И живот дальше нужно расслабить. Здесь хитрость в том, что мы
вроде бы его втянули, чтобы толкнуть поясницу, так вот --- расслабить живот нужно внутри,
то есть мышцы
живота...\ можете попробовать там подавить на мышцы живота сбоку как-то или просто пошевелить
живот. Если у вас там перенапряжённый живот, то у вас он совсем шевелиться не будет. Как бы вы его
там ни двигали, он будет как такая плита. Так вот, он должен быть как корзина с бельём, образ мне
нравится. Бельё, наверное, все стирают. Корзина с бельём или там таз, такой не жёсткий, а
пружинящий таз из пластика, например, который сверху накрыт крышкой. Соответственно, дно таза
или дно корзины с бельём это поясница, борта таза это таз и сверху рёбра и боковая часть
живота, а крышка --- это вот это вот собирание пресса. То есть живот не вываливается вперёд,
бельё не вываливается из корзины. Мы как бы собрали бельё и немножко утрамбовали крышкой, чтобы
оно не торчало наружу. Важный момент --- это то, что неважно, какое у вас...\ короче,
несущественно какой у вас объём живота. То есть, вы можете быть полными. Китайцы есть вообще, с
нашей точки зрения, страдающие ожирением, прекрасно занимаются Тайцзи, стоят столбом,
практикуют и живут там до глубокой старости. То есть в Китае, в принципе, нет подгонов по
поводу того, что нельзя быть толстым. Можно, если вы чувствуете себя хорошо и как бы у вас всё
функционирует. Но, тем не менее, и то же самое не проблема, если вы худые. Главное, чтобы мышцы
работали, чтобы вы себя хорошо чувствовали и так далее. Так вот, выстроили поясницу, живот.

Значит, дальше нужно расправить позвоночник. Позвоночник проще всего расправить, чуть-чуть
потянувшись макушкой вверх с одной стороны, а с другой стороны расслабив ноги и провалив таз
вниз. Позвоночник --- это ниточка, которую мы растягиваем в две стороны --- вверх-вниз. Значит,
плечи расправлены, то есть мы их не втягиваем, не сжимаем, но при этом верхняя часть спины должна
быть чуть-чуть округлой, сутулой, чуть-чуть. Сутулость должна быть по линии горизонтальной. То
есть сутулость достигается не за счёт того, что мы сгибаем позвоночник, а за счёт того, что мы
чуть-чуть подаём плечи вперёд. Если смотреть сверху, то корпус человека, практикующего столб
или практикующего Тайцзи --- это линза. Спереди чуть-чуть выпуклая за счёт естественной
выпуклости грудной клетки, а сзади сильно выпуклая за счёт округлости, за счёт вот этого
подавания плеч вперёд. Значит, плечи при этом опущены вниз. Опять-таки, если это непонятно,
можно сначала сознательно потянуть плечи вверх, как бы толкнуть, поджать плечи к ушам и потом
просто на выдохе опустить. То ощущение, которое у вас возникнет, когда плечи упали вниз, вот
оно должно быть постоянным. Значит, мы практически уже закончили. Макушка вверх, затылок
чуть-чуть назад. То есть, мы не просто позволяем голове всплыть, и тем более мы не тянем силой
голову вверх. Если голову тянуть, то шея будет напряжена, и это вредно для дыхания, для
кровоснабжения и так далее. То есть, просто голова вот как воздушный шарик всплывает вверх. И при
этом затылок чуть-чуть подан назад, подбородок опущен вниз. Вот у нас, значит, вот это столбовое
состояние, столбовая стойка.

Обратите внимание, я почти ничего не сказал про руки. Почему?
Потому что положение рук может быть разным. Но если у вас всё время будут плечи опущены вниз,
то независимо от положения рук, руки будут свободные и раскрепощённые. И наоборот, как только вы
плечи поднимаете вверх, то дальше как бы вы ни пытались руки расслабить, руки расслабляться не
будут. Мы говорим о трёх корнях в теле. У нас есть корень ног\'и --- это бедро. Чтобы ноги
работали правильно, вот мы говорили, чтобы колено было на месте, нужно выстроить таз и бедро.
Корень рук\'и --- это плечо. Поэтому если мы хотим, чтобы руки работали правильно, нужно плечи
расслаблять и выставлять на место, ну и опускать вниз. И корень всего тела --- это поясница.

Но я сразу скажу, первые 2--3 года пытаться полностью поставить поясницу на место в движении
бесполезно. Ну это я к тому, что если вы тренируетесь 2--3 года и у вас нет ощущения, что у вас
поясница на месте, учитель говорит, что она ещё не на месте --- не переживайте, не
расстраивайтесь. Это нормально. По моему опыту проще всего поставить ноги и руки. Ноги и руки
люди ставят за несколько месяцев. Поясница, постановка поясницы занимает несколько лет. Это при
наличии нормального учителя. Без учителя человек может всю жизнь заниматься и будет в наивной
уверенности, что у него поясница работает, а там на самом деле всё плохо. Поэтому здесь
конечно как бы я сейчас подробно и красиво, и вроде бы правильно ни объяснял, без
наличия мастера, без наличия наставника, в точности поставить всё тело на место почти нереально.

Ну, значит, вот мы с вами прошли всё тело снизу вверх. Давайте как раз посмотрим, что у нас
там, какие вопросы есть. Так, значит: ``Какие-то особые рекоммендации по постановке стопы для
людей с плоскостопием?'' Нет, никаких особых рекомендаций не надо. Значит, стопа с плоскостопием
--- это стопа, где мышцы работают не правильно, то есть слишком расплющенная стопа.
Соответственно, когда вы начнёте регулярно стоять столбом, у вас мышцы стопы начнут
активизироваться, и естественным образом плоскостопие уйдёт. С ним не надо ничего делать. Ну,
если какая-то запущенная стадия плоскостопия, то можно, конечно, носить какие-нибудь
ортопедические стельки, ещё что-то, но достаточно просто взрослому человеку несколько лет
позаниматься Тайцзи, а ребёнку, ну и там подростку лет до 12--14 достаточно полгода
позаниматься Тайцзи, и плоскостопие уходит. Это я могу уверенно говорить, я обучал детей и
значит у меня у самого в детстве было плоскостопие. Это очень легко всё уходит.

Значит, что
касается того, что навредит: нагрузка, которую человек получает в столбе, это практически та же
самая
нагрузка, которую человек получает просто при ходьбе. Ну, если человек
опасается, самый простой способ --- просто позаниматься с наставником и хотя бы там, не знаю, если
нет возможности постоянно ходить к учителю, то хотя бы раз в месяц или там раз в два-три месяца
появляться на тренировке, чтобы вам поправили столб и показали, как всё должно работать.
Что касается прям, чтобы навредить от регулярных занятий, нет. От регулярных занятий навредить
невозможно. Китайцы говорят, что в принципе, если вы стоите на ногах, то вы можете
тренироваться и будет польза. Это важно.

Ну, идём дальше. Вот мы выстроили столб.
Возникает вопрос: а при чём здесь боевые искусства? То есть, если человек стоит в столбе,
то какое это имеет отношение к боевому искусству, потому что в поединке или в сражении, в бою
нужно двигаться. Мы не говорим про оздоровительный эффект столба, это прям отдельная история.
Мне кажется мы запланировали там лекцию, посвящённую оздоровительным аспектам тайцзицюань,
прямо отдельно будем разговаривать. Сегодня говорим именно про движение тела, механику тела в
тайцзицюань. И с точки зрения механики столб это каркас, на который добавляется какое-то
движение. Можно провести аналогию с автомобилем.
Если у вас есть автомобиль или если вы разбираетесь в автомобилях, то вам это,
наверное, будет лучше понятно. Если нет, ну, тогда нужно искать другие какие-то аналогии. Вот
аналогия с автомобилем. У автомобиля есть несущая конструкция. Это может быть рама, если
это рамный автомобиль, либо несущая конструкция --- кузов. Значит и кузов и шасси. Вот если у вас
нарушена несущая конструкция, то есть если у вас повело кузов, если у вас смещено шасси, то как
бы вы ни пытались чинить всё остальное, ваш автомобиль будет постоянно не просто плохо ездить,
он будет ломаться. Просто от того, что вот эта базовая конструкция нарушена.
Второй пример с домом. Но опять, не знаю, если вы
живёте не в своём доме, вам это, наверное, не очень понятно, но вот люди, которые живут в своём
доме и которые строили свой дом, ремонтировали, эти вещи должны понимать. У нас в доме есть
фундамент. Если вы сэкономили на фундаменте, то через несколько лет, или через 10 лет, но рано или
поздно это аукнется. То есть, фундамент начнёт плыть,
в доме пойдут трещины и прочее прочее. Соответственно, нужно вложиться сначала, когда
строим дом --- сделать хороший фундамент, потом вывести стены, крышу. А потом вот всё
остальное... Когда у вас есть фундамент, стены и крыша, дом можно уже обживать, уже можно там делать
отделку, уже можно делать отопление и прочее прочее прочее. То есть вот это принципиально.
Точно
так же человеческое тело. Мы выстраиваем сначала несущую опорную конструкцию, и это как раз
столбовая практика. А дальше мы, например, начинаем ходить. Когда мы ходим, у нас есть позы,
где у нас вес равномерно на две ноги. У нас есть позы, где вес переносится с ноги на ногу. У
нас есть позы, где вообще одна нога в воздухе, и даже у нас есть позы, где мы, например, в
прыжке. Ну, не позы, но практики.

Прыжки в тайцзицюань используются
очень активно. Значит, если вам где-то кто-то рассказывает, что в тайцзицюань нет прыжков, чего
там ещё нет в тайцзицюань: нет ударов, нет блоков, всего нет, короче, только на духовной
энергии вы побеждаете противников, а ещё лучше вообще никого не побеждать, просто противников
не существует и прочее прочее. Это всё понятно, что болтовня как бы. Но это болтовня, которая
людям дурит голову. Я как профессиональный психолог прекрасно понимаю, что людям голову
задурить несложно, особенно если они этого сами хотят. Но давайте исходить из того, что
ещё раз, это боевое искусство, а значит нам нужно уметь прыгать и бить ногами, и значит,
уворачиваться --- с помощью движений каких-то, с помощью прыжков, пируэтов --- от атак, которые на
нас направлены и прочее прочее. То есть это вот. Ну, любой фильм хороший про ушу посмотрите --- и
вот, пожалуйста, всё там есть. Посмотрите фильмы про Тайцзи, там, с Ли Лян Цзэ, например, или там
классические китайские фильмы про историю тайцзицюань, там где Ян
Лучан молодой --- ну вот там всё, всё есть. Там прыжки есть, удары есть и всё есть. И разговоры о
том, что этого в Тайцзи нет, потому что это какое-то древнее священное искусство, которое основано
на духовной энергии --- болтовня. Люди, которые так говорят, ничего не понимают. Либо им задурили
голову, либо они сознательно заблуждают.

Движок Тайцзи предполагает, что всё, что мы делаем, мы
делаем той или иной вариацией вот этой столбовой позиции. Значит, а теперь вопрос: причём здесь
центр тяжести? Хорошо, вот мы стоим в столбе -- причём здесь центр тяжести и как он влияет на
наше перемещение? Дело в том, что центр тяжести... У мужчин он чуть выше, где-то в районе пупка,
но опять, в зависимости от конституции мужчины. У женщин он чуть ниже, ближе к тазовым костям, но
в любом случае центр
тяжести находится где-то в интервале между пупком и тазовыми костями. То есть, обычно говорят,
что область Даньтянь, значит, нижнее киноварное поле, оно находится на 3--4 сантиметра ниже пупка.
То есть, если вы, например, положите себе ладонь горизонтально на живот, так чтобы большой палец
лежал на пупке. Вот там, где у вас будет мизинец, это как раз область нижнего киноварного поля.
И это
тот центр, про который мы говорим, на который мы медитируем, в который мы погружаем внимание,
направляем, и так далее, и так далее. Для чего? Ну,
во-первых, для того, чтобы... Ну, это в некоторых учебниках встречается, и я согласен с этой
точкой зрения, она имеет смысл: для того, чтобы у нас внимание равномерно распределялось по
телу. То есть, когда у нас внимание находится в центре тела, то нам проще направить внимание и
в ноги, и в руки, и в макушку, везде. Если у нас, например, внимание в сердце или тем более в
голове, если мы наблюдаем за миром, за своим движением из глаз, то чисто геометрически нам
сложнее будет осознавать ноги, землю под ногами, то, что происходит там где-то внизу. И поэтому
будут сложности с равновесием, с координацией, с перемещениями и прочее, прочее.

Значит,
соответственно, мы своё внимание постоянно перемещаем в центр.
Мы можем сказать, что
центр --- это место не просто где у нас внимание, центр --- это место, откуда начинается
движение и куда оно возвращается. То есть, любое движение тайцзицюань начинается из центра,
по крайней мере, так полезно думать. Есть всякие нюансы, есть какие-то тонкости, есть движения,
которые начинаются из пальцев. Есть движения, которые начинаются из макушки. Есть, да. Есть
движения, которые начинаются от стопы. Но это уже какие-то детали и тонкости. Для начала
полезно представлять себе, что всё, что мы делаем, мы делаем из центра, из поясницы. То есть
вот эта область под пупком, значит, внутри живота, между брюшиной и позвоночником. Вот оттуда
у нас начинается движение. Опять-таки, как это нарабатывается? Это очень просто нарабатывается:
любое движение, которое вы совершаете в тайцзицюань... Если у вас одна рука свободна, если вы
делаете какую-то технику одной рукой, то свободную руку вы кладёте либо спереди на низ живота,
под пупок,
либо сзади на поясницу. На поясницу, можно даже немножко ниже на крестец. Вот. То есть
фактически наша рука --- это наш первый и самый главный тренер. Почему? Потому что в руке много
внимания. Мы с детства привыкаем что-то руками делать, поэтому где у нас рука, там у нас есть
какое-то количество внимания. Поэтому, например, движения, базовые техники Тайцзи --- шёлковая
нить «чань-сы»,
когда наматываем рукой, рисуем круги, спирали и так далее --- свободная рука лежит на животе или
на пояснице. Значит, если мы двигаемся без рук, а в тайцзицюань есть упражнения, где мы
перемещаемся без рук ---
отработка шагов каких-то или что-нибудь такое, то тогда обе руки могут лежать внизу живота или
там на пояснице или одна рука спереди, другая сзади. Опять-таки вариантов очень много. Один из
наших хороших друзей и коллег, Олег Олегович Коровин --- Олег, привет, если ты слушаешь эту
лекцию --- он, например, регулярно, когда стоит в базовой стойке, столбовой стойке Тайцзи, он,
например, держит руки на бёдрах в районе точек куа. То есть, пожалуйста, это практически та же
самая область нижнего киноварного поля, но руки чуть шире, не по осевой линии, центральной,
а чуть вправо-влево на точки куа. Точки куа --- это внутренняя проекция тазобедренных суставов,
проекция на паховую складку. Скажем так, спереди на паховой складке, где-то на уровне
верхнего края лобковой кости. Вот. То есть руки могут лежать на бедрах, руки могут
лежать на пояснице, руки могут лежать на животе. Таким образом, мы постоянно собираем внимание
вот в этом нижнем центре, и все движения, которые мы делаем, мы делаем оттуда и туда.

Это не просто философия, я бы даже сказал, это никакая не философия, это прямая инструкция, как
необходимо двигаться в тайцзицюань. Движение начинается из центра и движение должно закончиться
в центре. Ещё иногда говорят, что оно должно закончиться в земле, но опять-таки это уже
какие-то нюансы. Через центр движение в конце должно пройти. Любое техническое действие,
которое вы выполняете в тайцзицюань, вы выполняете с соблюдением этого правила. То есть, если
вы, например, делаете шаг...

Вот рассмотрим простой пример, шаг ногой. Во время шага ногой мы
сначала подтягиваем одну ногу к себе, сначала подтягиваем чуть внутрь, потом поднимаем ногу вверх,
как будто хотим втянуть пятку себе
в промежность, и стопу себе в промежность. После этого... Вот у нас --- мы собрали ногу в центр,
после
этого у нас нога выходит туда, куда мы хотим шагнуть: вперёд, назад, вправо, влево или обратно
туда, откуда мы начали движение, потому что в тайцзицюань есть такое правило «5 шагов», то есть
шаги вперёд-назад, вправо-влево и ещё шаг на месте, это тоже шаг. Соответственно, вот мы
втянули ногу в центр и только потом она куда-то пошла. И когда она пошла, то шаг заканчивается
после того...\ не когда нога опускается на землю, а когда вы обратно собираете внимание в центр
тяжести, в эту область Даньтянь.

То есть, ещё раз, цепочка, последовательность выполнения шага: переместить внимание в ногу,
подтянуть ногу к себе, подтянуть ногу к тазу. Можно достаточно высоко ногу поднимать. Ну,
минимум так, чтобы колено было на уровне бедра, иногда и выше поднимается нога.
После этого нога идёт туда, куда мы хотим сделать шаг, и завершается шаг перемещением внимания
обратно в центр. Вот это базовая цепочка действий, чтобы просто шагнуть. Кажется --- очень сложно,
да? На самом деле, когда вы это тренируете каждый день по 10--15 минут, то через год вы это
делаете практически не задумываясь. И в своё время мы это много обсуждали, что как же так, это
ходить неудобно. Ну нет, ходить не просто удобно, ходить очень быстро, легко и ноги не
устают. Поэтому любое движение, которое мы выполняем...\ удар ногой --- всё то же самое: мы
втягиваем ногу в центр, дальше наносится удар ногой, внимание опять собирается в центр. Нога
при этом может остаться в воздухе. В тайцзицюань есть удары ногами, когда нога не опускается
вниз. Нога может, значит, опуститься в том направлении, в котором мы делали удар, а может
вернуться обратно. Это неважно. Важно то, что внимание вы всегда собираете в центре. То есть
пока вы внимание не собрали, движение не закончено. Вот это требование работы центром в движке
Тайцзи. Значит, соответственно, движение рукой. Вы начинаете движение рукой с того, что
собираете внимание в центр, выполняете движение рукой, это может быть толчок, удар, маховое
движение, какое-то втягивающее движение и прочее, но в конце либо руку вы подтягиваете к своему
центру тяжести, либо рука остается где-то там, внимание собирается в центр. Значит, опять-таки
в парной работе в чём это проявляется? Это проявляется в том, что тактически, вот если
рассматривать именно тактику взаимодействия с партнёрами, с оппонентами в тайцзицюань, то
тактически мы партнёров, оппонентов связываем, скручиваем, лишаем подвижности, либо
совершаем различные травмирующие
воздействия по суставам, то есть ломаем, то есть опять-таки движение, как будто мы ломаем палку или
перекручиваем, передавливаем верёвку, завязываем узлы на верёвках и так далее. И здесь всё то
же самое, движение сначала направлено на партнёра-оппонента, но в конце движения мы собираем
внимание и часто руки к себе в центр. В этом смысле, когда описывают метафоры,
метафорами описывается тайцзицюань как состояние и как некая активность динамическая, то часто
говорят такую метафору как дыхание, просто дыхание. Когда человек дышит, он сначала что-то
втягивает, потом выпускает, потом опять втягивает, выпускает. Вот это цикличное движение. И
втягиваем мы в себя, выпускаем из себя. Вот только когда мы втягиваем и выпускаем воздух, то у
нас внимание может быть где-то во рту, в горле, ну максимум в грудной клетке,
в диафрагме. А в
тайцзицюань в этом смысле нужно дышать...\ иногда мастера Тайцзи говорят «дышать телом».
То есть мы
внимание выпускаем из центра тяжести и обратно в центр тяжести собираем.

Фактически
то, что я описал... То есть ещё раз: мы сегодня проговорили что такое столбовая структура, мы
проговорили что такое центр и как принцип вот этого центра, центра тяжести, нижнего Даньтянь,
киноварного поля, как это реализуется в движении. Фактически вот на этом построен весь движок
Тайцзи. Есть следствия из этих двух моментов, то что мы в столбе и то что мы постоянно
работаем через центр. Ну например, какие следствия? Например, правило, что в тайцзицюань
никогда не бывает веса на двух ногах за исключением позы медитации, ну, собственно, начальная
поза --- «тайцзи», стойка великого предела, и позы завершения движения, когда мы стоим в «уцзи», то
есть это поза пустоты, беспредельности, когда ноги вместе. Вот это две позы, где вес распределён
поровну. Все остальные позы предполагают перенос веса, предполагают асимметрию, то есть вес не
должен быть на двух ногах. Есть даже такая жёсткая пословица, что у даоса или у мастера
Тайцзи, ну это я слышал в разных вариациях, вес на двух ногах только один раз в жизни --- в гробу.
Когда человек умер, его кладут в гроб, вот у него там вес на двух ногах.
Ну, это метафора. Имеется ввиду,
что только в гробу вы полностью в покое, ваше тело в покое, оно никуда не дёргается, не
перекатывается и так далее. Живой человек находится в постоянном микродвижении, вот это важно.
На самом деле, даже когда мы стоим на двух ногах, мы постоянно чуть-чуть качаемся с ноги на
ногу, просто мы этого не замечаем, мы привыкли к этому.

Значит, ещё какие требования? Ну,
например, требование, чтобы было согласованное движение рук и ног. Ну, это опять-таки
требования, происходящие из того, что у вас постоянно работа вот с этим центром, с нижним
Даньтянь. Значит, если у вас внимание в Даньтянь, то у вас руки и ноги будут как бы равномерно,
согласованно двигаться. В тайцзицюань есть правило, что если одна часть тела двигается, то
двигается всё тело. И если одна часть тела остановилась, то всё остановилось. То есть у нас любое
движение выполняется согласованно во всех частях тела. Меньше всего у нас работает макушка,
физически, то
есть активность макушки такая, физическая --- минимальная. Макушка просто находится в покое. Всё
остальное тело может двигаться очень активно, но эта активность сбалансированная,
согласованная. Есть правило, но это опять-таки уже требования для таких более подготовленных
людей, что руки и ноги должны быть увязаны в суставах. То есть, увязанность в суставах или
связанность в суставах --- это как будто...\ аналогия --- как будто у вас суставы перебинтованы
тугими
бинтами. Можете взять, попробовать. Возьмите эластичные бинты, сбинтуйте себе основные суставы:
голеностопы, колени, поясницу, плечи, локти, запястья. Вот запомните ощущение. То есть оно как-бы
вроде бы скованность с одной стороны, но эта скованность от непривычки. Профессиональные
спортсмены играют в бинтах, перебинтованные практически везде, где только можно, чтобы не
мешало двигаться. И при этом никакой скованности нет. С одной стороны. А с другой стороны --- это
чувство защищённости. То есть суставы, для чего спортсмены себя бинтуют --- суставы
забинтованные, у них есть опора, некие каркасы, простейший экзоскелет. Но только в тайцзицюань
мы это делаем без всяких бинтов, без всяких усилителей и так далее. Значит, любая защита в
тайцзицюань --- это то, что скорее лишает наработки правильной техники, нежели помогает. С
защитой вообще всё очень сложно. Если мы, например, проводим какие-то спарринги или парную
наработку ударов --- вот у нас сегодня на семинаре как раз стучали друг по другу --- то желательно
это всё делать без защиты, то есть лучше делать менее жёсткие удары, но чтобы вся техника
выполнялась без защиты, потому что защита физическая, внешняя искажает ощущение тела. Ну, по
крайней мере, вот как говорят мастера, которым я доверяю. Значит, в принципе, наверное, всё.
Ну, я мог ещё какие-то мелочи упустить. Здесь я вижу нескольких инструкторов сегодня на лекции.
Если я что-то упустил, какие-то
рекомендации по движению в тайцзицюань, можете написать. Но по-моему, основное всё сказал. Вот.

Опять-таки, вот то, что мы сегодня обсуждали, я могу честно признаться, что не во всех
учебниках по тайцзицюань это есть, потому что это...\ кому-то это неочевидно, кому-то это кажется
каким-то раскрытием секретных секретов там и так далее и так далее. ``Лучше пусть умрёт боевое
искусство чем передать его недостойным,'' как раньше говорили. Ну, на мой взгляд это уже давно всё
устарело и в этом нет никакого смысла. Какое там секретное боевое искусство когда тут же на
соседней, как говорится, улице, в соседнем городе мастер который всё это направо и налево
преподаёт, лишь
бы приходили ученики и занимались. То есть, соответственно, вот то, что мы сегодня разбирали,
кто-то в принципе не знает, не понимает, а кто-то знает, понимает, но как бы головой. Тоже
такой момент, что то, что я сегодня рассказывал, я уверен, что кто у нас занимается несколько
лет, всё это слышали. Проблема в том, чтобы это реализовать в движении. То есть проблема не в
том, чтобы это, ``Ага, я это понимаю.'' Ну, хорошо, что вы это понимаете, а сделать-то можете? То
есть всё, что мы сегодня перечисляли --- это прямые требования к движению в тайцзицюань. До тех
пор, пока вы это
не реализовали в своём движении, не надо говорить ``Я понимаю.'' Это прямо вредно. Понимаешь...
Как говорил один из моих учителей «Знать --- значит делать». Знаешь, понимаешь --- покажи.
Не можешь показать --- не болтай. Ну вот, наверное всё.

Я с удовольствием отвечу на вопросы. Кто хочет,
пишите вопросы в чате. Кто не хочет, можете... Ну, кто хочет вслух спросить --- поднимайте руку, я
дам вам микрофон, и, соответственно, можно будет задать вопрос. Так, ``Стопы в высоком, в низком
столбе куда смотрят?'' Это уже детали. Стопа должна быть согласована с коленом. Вот это строгое
требование. Если у вас стопа смотрит вперёд, а колено куда-нибудь вбок, это зло. То же самое,
если стопа смотрит вбок, а колено вперёд, тоже зло. Стопа и колено согласованы. Дальше, вот
ещё раз, мы же говорили не просто про столб, а про перемещение в столбе. При перемещении стопы
крутятся туда-сюда. Опорная нога почти никогда не крутится, хотя если мы делаем прыжки или
какие-нибудь пируэты на одной ноге, то там и опорная нога может вращаться. А, соответственно,
подвижная нога --- в тайцзицюань есть опорная нога и подвижная --- подвижная нога может вертеться
вообще куда угодно. Поэтому когда столб... Ещё раз, столб --- это какая-то совсем базовая база, от
которой
мы уже двигаемся. Но в высоком столбе можно сказать, что да, лучше вперёд, причём не просто
вперёд, а пятки чуть шире, чем носки, потому что стопа у нас не прямоугольная, стопа у нас --- это
трапеция, поэтому вот параллельность стоп выставляется по внешнему ребру стопы. В этом случае
у вас носки будут чуть уже, чем пятки. Пятки ставятся в высоком столбе по ширине плеч, а
носки, соответственно, чуть-чуть смотрят внутрь. А в низком столбе, ещё раз, очень сильный
вариатив.

Так, ещё вопросы. ``Столб больше самостоятельная практика или ей должно быть выделено
время на тренировке?'' Ей должно быть выделено время на тренировке вашей личной. То есть, в вашей
личной тренировке. Первые три года я рекомендую стоять столбом каждый день, хотя бы по 10--15
минут, лучше побольше. Ну, не обязательно каждый день.
Например, можно каждый день по 15 минут, и раз в неделю или раз в две
недели стоять столбом до упора. То есть 20, 30, 40 минут, сколько выстоете. А на групповой
тренировке столб --- это элемент обучения. То есть там нужно не просто стоять, там нужно
править себя в столбе. То есть в этом смысле есть столб как упражнение, есть столб как
методика тренировки. Это разные вещи. Есть ещё отдельно столб, как медитативная практика. Вот
это мы с вами будем разбирать. У нас отдельная лекция, что такое упражнение
в тайцзицюань, что такое
медитация и что такое методика тренировок. Столб --- и то, и другое, и третье, в зависимости от.
Соответственно, если учитель, который вас обучает, понимает эти вещи, то он методику будет
использовать на групповом занятии. Просто так стоять столбом на групповом занятии --- тратить
время. Зачем? Это лучше делать дома. Вот, как-то так.

Ещё вопросы? Георгий, ты там руку
поднимал, если что --- ещё раз можешь поднять. Приятно, что много людей слушает лекции, причём с
каждой лекцией всё больше и больше. Кто не успевает слушать лекции в прямом эфире, мы все
записи выкладываем. Смотрите внимательно, у нас выкладывается запись в Вконтакте, в Телеграме.
По-моему, была ещё идея выкладывать где-то в Рутубе, но я не помню, Лера смогла это реализовать
или нет. В принципе, все записи у нас есть, и после каждой лекции приветствуются вопросы,
какие-то уточнения, что-то ещё. Значит, у меня спрашивали, какая программа лекций. Я могу
сказать на следующие несколько лекций программу. Дело в том, что я прекрасно понимаю, что может
всё пойти не по плану. У меня план расписан до конца, всё может пойти не по плану. По плану у
нас следующие лекции --- это Великая Триада, значит, Земля-Небо-Человек и, соответственно,
внутренняя работа в столбе, и шаг Тайцзи. Вот это ближайшие две лекции, которые мы будем
разбирать. Ну вот сегодня мы про шаг уже говорили, соответственно, это будет подробнее, и про
столб мы говорили, тоже это будет подробнее. То есть,
будем ещё раскрывать дополнительно всю эту кухню. Вот, ну если больше вопросов нет, то
на сегодня заканчиваем. Всем большое спасибо, что вы сегодня слушали, спасибо всем, кто задавал
вопросы, и увидимся через неделю примерно в это же время.
\bye
