%&12pt
\pdfpagewidth=297mm
\pdfpageheight=210mm
\pdfhorigin=1in
\pdfvorigin=0pt
\input QUIRE
\shhtotal=\pdfpagewidth
\htotal=.5\shhtotal
\vtotal=\pdfpageheight
\shoutline=0pt
\shstaplewidth=0pt
\shcrop=0pt
\shfootline={}
\shthickness=.27mm
\qtwopages \shipout\vbox{}

\horigin=10mm
\hsize=\htotal \advance\hsize by-2\horigin
\advance\hsize by-\QUIRE
\output={\ifodd\pageno\else\hoffset=\QUIRE\fi \plainoutput}

\vorigin=7mm
\vsize=\topskip \advance\vsize by37\baselineskip

\footline={\raise3pt\line{\hss\tenrm\folio\hss}}

\hyphenation{тай-цзи-цю-ань}

Добрый вечер, дорогие друзья! Продолжаем говорить о 
тайцзицюань.  И сегодня темой нашей лекции будет 
вопрос преподавания тайцзицюань. Я немножко напомню,
о чём мы говорили на первой лекции. Мы говорили о том,  
что тайцзицюань --- это классическое китайское боевое
искусство,  предполагает одновременную реализацию 
трех задач, трех целей в развитии, в тренировке. Первое 
--- это, собственно, боевое искусство.  Второе --- это... 
скажем так,  духовная практика, духовное саморазвитие. 
И третье --- это оздоровление,  здоровье.

Вопрос в том, что когда тайцзицюань создавалось, это было ещё
раз... конец XVII, XVIII век, была одна ситуация,  скажем так, 
общественно-политическая. Когда мы сегодня преподаём 
тайцзицюань и изучаем тайцзицюань, ситуация 
совершенно другая.  Фактически у нас возникает 
несколько проблем, когда мы изучаем тайцзицюань и 
преподаем его. Прежде всего, это проблема культурного 
перевода.

Многие понятия системы тайцзицюань ---  это 
понятия не просто из китайской культуры, это понятия
из традиционной китайской культуры,  которая уже в 
самом Китае нуждается в комментариях, в объяснениях, 
в какой-то интерпретации и так далее.  То есть это 
метафоры,  образы,  какие-то иллюстрации, примеры, 
основанные на классической китайской литературе,  на 
китайской философии.  Я более чем уверен, что 
современные китайцы в массе своей плохо разбираются в 
этих вопросах. Я сам лично в поездках в Китай наблюдал 
ситуации,  вроде бы образованные китайцы, когда я 
называю какие-то факты из китайской истории, причём 
факты у нас общеизвестные, они делают удивлённое лицо 
и спрашивают, откуда ты всё это знаешь.  То есть как-бы 
то,  что у нас не особо скрывается и при минимальном 
интересе к китайской культуре и истории доступно, для них это 
что-то сложное, не то чтобы закрытое, но недоступное 
большинству населения.

Вот, поэтому даже в самом 
Китае, когда сейчас идёт преподавание тайцзицюань,  
возникает вопрос перевода. Но китайцы давно с этим 
феноменом столкнулись.  И Конфуция каждые там 100 лет 
заново комментируют,  переиздают,  интерпретируют и 
так далее. И Лао Цзы тоже самое и так далее. То все 
китайские классические тексты каждые 100 лет, ну может 
быть каждые 200 лет заново переиздаются, подвергаются 
какому-то перепросмотру и прочее, прочее.  

Соответственно,  когда это всё привозится к нам,  
возникает ситуация... Китайский язык очень 
образный.  Каждое китайское слово имеет за собой 
несколько десятков смыслов.  Для сравнения, русский 
язык тоже богатый язык, никто не спорит, и я очень 
люблю русский язык. Но в русском языке в среднем за 
одним словом стоит семь-восемь смыслов. Это 
достаточно много, но...  В классическом китайском языке 
за каждым словом стояло 30--40 смыслов.  Это официальные 
данные лингвистические, есть специальные 
исследования на эту тему. Да, современный китайский 
язык попроще,  поэтому его называют упрощённым 
китайским языком.  Но тексты по тайцзицюань писали 
естественно на старом китайском языке, на Вэньяне.  Вот эта 
глубина метафор, богатство метафор,  они оттуда 
пришли.

Соответственно,  это все адекватно, корректно 
перевести очень сложно.  Это первый момент. Второй 
момент --- это то, что многие метафоры,  когда
тайцзицюань создавался и преподавался в XVII--XVIII веке,
многие метафоры были рассчитаны, прежде всего,  на 
крестьянина. То есть Чэнь Цзя Гоу --- это деревня. Пусть 
не, скажем так, не рядовая деревня,  ну такое 
благоустроенное место.  Вряд ли наставник 
императорской гвардии жил бы в каком-то, скажем так,  
задрипанном...\ местечке. Да, то есть деревня Чэнь Цзя Гоу
место приличное.  Ну, какой-то местный центр. Но тем не 
менее это деревня. То есть, соответственно,  вполне 
вероятно,  первые ученики тайцзицюань,  первые 
поколения, может быть, даже первые 10 поколений 
учеников тайцзицюань были либо крестьянами, либо 
мелкие торговцы, чиновники. То есть люди не слишком 
глубоко погружённые в какую-то культурную глубину. 
Поэтому метафоры, вот если вы возьмёте классические 
тексты по тайцзицюань, например, песня о 13 позициях, 
наставление...\ о тайцзицюань и так далее,  они есть на 
русском языке, там вот видно,  что язык описания он 
рассчитан на такого не очень как бы образованного 
человека. То есть на человека,  который знает какие-то 
основы, ну азы, там, например, о натур философии что-то 
слышал  возможно, о воинском искусстве, там, возможно
что-то слышал, о том, как ведутся боевые действия. И 
очень много метафор таких бытовых. Например,  
классическая метафора, которая используется в старых 
текстах тайцзицюань --- это тележное колесо.  Понятно,  
что это преподавалось именно с расчётом на того, кто 
приходил на занятия. То есть это люди, которые вот 
умеют работать руками и хорошо понимают, как устроено 
тележное колесо.  Как минимум возникает вопрос, что 
если мы сейчас преподаем тайцзицюань, понимаем ли мы, 
насколько сильно изменился мир? То есть я могу 
сказать, что из ныне живущих людей... Я, например, 20 лет 
жил в деревне, я телегу видел. Вы знаете, сколько раз я 
видел телегу за 20 лет жизни в деревне? Три раза. Вы 
знаете, в современной деревне, чтобы встретить телегу, 
ну, это надо умудриться. Потому что люди в современной 
деревне на телегах не ездят. Там какой-то, может быть, 
один дедушка хардкорный на телеге сено возит, и всё, 
остальные...\ пользуются тракторами какими-нибудь, 
мото-блоками и прочее и прочее.

Значит, 
соответственно,  если мы хотим современному человеку 
объяснить принцип, как работает поясница,  а в старых 
текстах по тайцзицюань поясница сравнивается с осью 
тележного колеса, то мы должны объяснить работу 
поясницы на каких-то...\ понятных примерах. Ну не знаю,  
говорить о том, что поясница это подшипник. С 
подшипниками, ну по крайней мере, мужчины скорее всего 
сталкивались. Мужчины, которые имеют некоторое 
отношение к механизмам, к инженерному, скажем так, 
делу, может быть, занимались каким-то ремеслом и так 
далее.  Ну, в конце концов,  владелец автомобиля, скорее 
всего, знает, что такое подшипник. Вот. И вот если вы 
будете ученикам говорить, что поясница --- это 
подшипник, на котором крутится тело, то это хоть как-то 
будет понятно.

Но ось тележного колеса --- это всё 
равно,  что, я не знаю,  мифический зверь Кунь-Лунь.  В 
текстах о тайцзицюань, по-моему, мифический зверь 
Кунь-Лунь не используется, а в других... китайских 
текстах используется. И пока вам не объяснят, что это 
за зверь и зачем он вообще нужен,  то вы не будете 
ничего понимать. И эффективность изучения,  смысл
чтения книг, в которых написаны подобные фразы...  И, 
соответственно, эффективность обучения на основе 
таких книг очень сильно снижается.  Вот это то, чего 
многие переводчики и преподаватели тайцзицюань не 
понимают. Я встречал весьма образованных людей.  Будем 
откровенны, тайцзицюань в принципе образованные люди 
преподают и, как правило, изучают образованные люди. 

Значит,  вот мне просто любопытно. Буду благодарен, 
если вы в комментариях  под этой записью, не знаю, где 
вы её слушаете, Вконтакте, в Телеграме, ну, сейчас в 
Телеграме, но дальше мы её будем выкладывать на разных 
платформах. Напишите,  в комментариях,  какое у вас 
образование и кем вы работаете. Просто для такого 
экспресс-социологического исследования. Скорее 
всего,  у большинства тех, кто интересуется этой идеей, 
высшее образование.  Вполне возможно есть 
автомобиль или вы представляете себе, что такое 
автомобиль и каждый день вы имеете дело с какими-то 
механизмами. Но с тележным колесом вы никогда в жизни 
не сталкивались, с большой вероятностью.

Это просто 
пример.  И таких примеров очень много.  Вот, то есть ещё
раз, вот этот культурный перевод, это не просто, что 
вот есть китайская культура, есть русская культура.  
Это всё гораздо серьёзнее. Есть китайская культура 
фактически позднего средневековья по нашим аналогиям 
или эпохи Возрождения,  есть аналог европейского 
Ренессанса. Есть современный человек 21 века, который 
живёт в мире электроники, мире самолётов, ракет и 
прочее, прочее. Вот это проблема, что перевод и 
исторический,  и культурный.

Вторая проблема 
преподавания тайцзицюань в том,  что когда создавалось 
это боевое искусство, собственно, ни у кого не было 
вопросов. Зачем нужно боевое искусство?  Во-первых,  
в китайской традиции боевые искусства были, как 
минимум,  последние полторы, а по некоторым данным две 
тысячи лет. Это тоже часть восточного менталитета.  
Понятно, что воинские традиции были во всех культурах, 
но именно в восточных культурах воинские традиции 
тщательно собирались, анализировались,  записывались, 
 оформлялись в определенные трактаты, учебники и 
прочее, прочее.  И на сегодняшний день мы по истории 
китайских боевых искусств, скажем так, то, что 
написано...\ до начала 20 века, до начала эпохи массового 
книгопечатания в Китае, только до начала 20 века мы 
имеем...\ ну вот я сейчас могу ошибиться,  порядок 
величины, скажу, порядка 100 трактатов,  плюс-минус, то 
есть может быть чуть меньше, но не намного. То есть это 
не единицы,  это многие десятки трактатов по боевым
искусствам.

Посмотрите,  для сравнения европейские 
трактаты по боевым искусствам. За всю историю Европы 
написано было,  ну, хорошо, если 10 трактатов по боевым 
искусствам.  Да, и отдельная проблема, что в России 
трактаты по боевым искусствам, если и писались, то всё 
это писалось на берестяных грамотах. Соответственно, 
оно никак не сохранилось. Вот поэтому мы сейчас можем 
только гадать, какие были боевые искусства у нас на 
Руси в древние времена и так далее.  Данных по этому нет. 
Только какие-то мифы, фантазии и прочее, прочее.

Соответственно  для китайца боевые искусства --- это 
часть как бы естественного уклада. Но при этом надо понимать, что
такое естественный уклад. Боевые искусства --- это не 
то, что связано напрямую с армией. Боевые искусства --- 
часть определённого добродетельного поведения и 
добродетельной жизни.  Есть определённые 
конфуцианские добродетели.  В рамках этих 
добродетелей предполагается, что человек, если он 
идёт вот этим путём совершенного мужа,  то человек 
должен постоянно заниматься саморазвитием, 
самосовершенствованием. Соответственно, боевые 
искусства предлагали такой путь,  свой путь 
саморазвития, самосовершенствования.

С древних времён были мирские школы,  или, скажем, гражданские 
школы боевых искусств, были специализированные 
храмовые школы, монастырские.  Значит,  были военные 
школы, то есть такие более, как бы сказать,  более 
профессиональные, более...\ требовательные к какому-то 
оружию, которое необходимо использовать в реальном 
бою.  Понятно, что в реальном бою редко пользовались 
зубочистками, как мы сейчас говорим.

Вот 
классический китайский меч дзян,  которым любят 
фехтовать мастера Тайцзи.  Меч дзян в боевых конфликтах за 
последние полторы тысячи лет вообще никак не 
использовался, не употреблялся. Это было 
исключительно такое парадное церемониальное оружие.  
Ну и плюс им фехтовали аристократы, то есть это было 
такое аристократическое оружие.  Украшался дорогими 
всякими камнями и так далее. Ну то есть это не имело 
отношения к армии.  Армейские боевые искусства это 
было особое направление. Там, как правило, были 
простые какие-то средства. Есть там копье,  что-то типа 
алибарды и сабля со щитом.
Вот, соответственно, 
доводились до автоматизма приемы с этим оружием и 
прочее, прочее.

Были вот такие условно, ну, говорят, что 
два крупных направления. Это военная «унь» и гражданская 
«вэнь» направления. Ну и то и другое было боевыми 
искусствами. Монастырское оно скорее относилось к 
гражданскому, потому что монахи оружие, если 
использовали, то это были какие-то условно 
гражданские версии оружия. Вот. Вопрос в чём? Вопрос в 
том,  что китаец XVII, XVIII, XIX века он это всё воспринимал
как некое естественное поведение человека. Человек развивается, человек изучает 
конфуцианский канон,  человек занимается боевыми 
искусствами. Человек читает стихи, изучает 
классическую философию и так далее.

Отдельный момент 
--- это то,  что если мы спросим себя, какую роль играли боевые 
искусства в деле массового воспитания духа населения 
или какого-то массового оздоровления, то будет 
простой ответ --- никакого. Никакого. За всю историю Китая 
мы не имеем ни одного примера, когда бы боевые 
искусства использовались для массового оздоровления 
населения или массового обучения людей какому-то 
воинскому делу. Ну, за исключением, значит, этих 
сумасшедших товарищей из доводских сект, которые там, 
например, Восстание Желтых Повязок. То есть им нужно 
было быстро сколотить крестьянскую армию, вот для 
быстрой подготовки крестьянской армии, там, крестьян 
обучали каким-то примитивным приёмам, там, рубки 
саблей дау и уколов копьем.
 
Только в XX веке возникла идея 
массового оздоровления населения. Причём я уже на 
прошлой лекции говорил, что во многом эта идея пришла 
в Китай через Советский Союз и через Германию. То есть, 
значит,  благодаря Германии в Китае появились 
современные программы профессиональной подготовки 
военных, потому что до...  Скажем так,  до конца XIX, 
начала XX века в Китае не было какой-то единой системы 
подготовки военных.  Не было академии, не было каких-то 
специальных училищ и так далее.

Военных готовили 
сами военные. То есть, там была система военных 
поселений,  застав, гарнизонов, в которых как раз люди 
обучались. И китайцы, если кто не знает,  до 
середины XX века к солдатам относились как к людям, ну 
скажем так, второго сорта.
Потому что в 
классической китайской культуре есть такое правило, 
что из хорошего железа не делают гвоздей,  приличный 
человек не станет солдатом.  Да, то есть солдаты --- это 
было такое, ну как бы...  Ну в смысле человек, который 
никогда не станет на путь какой-то добродетели и 
прочее, прочее. Человек,  который не способен понять 
тонкости поэзии. Есть боевые искусства массовые, 
армейские.  Это было для каких-то типа примитивных 
людей.

Соответственно боевые искусства 
специализированные,  не армейские, есть там внутри 
закрытых школ, внутри сект. Они как раз могли 
быть элементом какого-то, как я уже говорил, 
саморазвития, духовной практики.  Но тогда это было не 
для всех, то есть тогда это было чем-то 
специализированным для узкого круга. Очень редко 
передавалось за пределы клана, рода и так далее.  Все 
эти истории, что клан Чэнь сохранял чистоту стиля и 
практически за пределы клана...\ прямые потомки Чэней,
многочисленные родственники и так далее. За пределы 
клана это все до, получается,  конца, до середины XIX 
века не передавалось.

Потом к ченям попал Ян Фу Хоу,  
он же Ян Лучан.  И отдельный вопрос, как он прошёл 
обучение у Чэней, но как-то прошёл. Скорее всего, 
приняли в клан, то есть ввели в семью, как бы усыновили, 
что называется.  Ну и, соответственно, он уехал в Пекин 
и начал преподавать свою версию тайцзицюань в 
Пекине,  которую мы сейчас знаем как стиль Ян.

То есть 
это я к тому,  что в китайской традиции боевое 
искусство не могло быть массовым. Ну, за исключением 
вот этой военной темы. Но военное боевое искусство, ещё
раз, было каким-то очень ограниченным в плане, ну,  
считалось, что там не может быть каких-то вершин.  Хотя, 
опять-таки,  были талантливые люди из, то есть мастера 
боевых искусств, представители армии. Тот же самый 
генерал Юэй Фэй, который считается практически 
небожителем,  таким полулегендарным покровителем 
боевых искусств,  и ему приписывают создание 
нескольких стилей боевых искусств и так далее.

Но это 
сильно раньше было,  чем создание тайцзицюань. Тем не 
менее,  тайцзицюань создавалось для такого очень 
узкого круга людей.  То есть предполагалось, что Тайцзи 
Цюань будет эффективным боевым искусством для 
конкретных задач. То есть это бой одного против толпы 
с элементами сохранения здоровья, с элементами 
продления жизни. Потому что хочется пожить подольше,  
потому что хочется передать свое искусство детям и 
внукам, а для этого тоже нужно пожить подольше и так 
далее, и так далее. Ну плюс вот эти духовные аспекты, 
потому что опять-таки гунфу, и соответственно нужно 
развиваться, нужно постигать какие-то философские 
истины и прочее и прочее. Вот. Значит,  здесь тоже есть 
проблема.

Вот смотрите, мы начали преподавание.  Я 
такую длинную, возможно, запутанную немножко сегодня 
беседу, веду.  У меня как бы идёт поток 
сознания,  но потом мы это все причешем и упорядочим. 
Соответственно, я что хочу сказать? Я хочу сказать,  
что вот обстановка, в которой создавался и развивался 
тайцзицюань,  никакого отношения к современной 
реальности не имеет. В современной реальности, ну вот 
буквально, что такое клан Чэнь?  Клан Чэнь --- это 
охранное агентство. То есть, если вы занимаетесь 
тайцзицюань, так как им занимались создатели этого 
искусства, Чэнь Вантин и его потомки, то вы должны быть 
профессиональными охранниками. Причём 
профессиональными охранниками там в каком-нибудь 
третьем, четвертом поколении. То есть ваш дедушка был 
охранником,  ваш папа был охранником, вы будете 
охранником, ваши дети и внуки тоже будут охранниками.  
Вот, то есть ЧОП какой-нибудь,  частное 
охранное предприятие. И так далее.

И тогда вот оно 
всё ровненько ложится. Но при этом вы будете такими 
охранниками, которые должны работать над собой, то 
есть не просто там стоять и с угрожающей мордой
следить, чтобы был порядок, а вы должны
работать над собой, то есть, возможно...\ представители 
вашего клана должны будут стать депутатами, поэтому 
надо изучать философию, психологию, политику,  
экономику и прочее, прочее.  Но при этом вот ваше 
ремесло, ваш заработок на протяжении нескольких уже 
столетий --- это вот охранное дело.

Представьте себе,  
насколько это сильно отличается от той жизни, которой 
мы сейчас живем. Это абсолютно невозможно сравнивать 
с тем, как мы сейчас живём.  То есть да, у нас есть люди, 
которые профессионально служат в силовых структурах,  
которые служат в армии, которые работают охранниками 
в полиции, например. Но во всех этих ситуациях Тайцзи 
Цюань даже если используется... У нас есть примеры в XX 
веке, когда тайцзицюань преподавался и охранным 
структурам,  и преподавался в полиции.  Ну, насчет 
армии не скажу, просто у меня нет данных. Скорее всего, 
в Китае, может быть, какие-то подразделения и 
проходили обучение мастеров тайцзицюань, у меня 
просто нет данных. Ну, про полицию есть точно данные и 
про охранные структуры и в Китае, и в России есть 
данные.

То есть всем остальным получается, вроде бы, 
тайцзицюань не нужен.  То есть боевое искусство не 
нужно. Почему не нужно? Потому что тайцзицюань --- это 
искусство не поединков. тайцзицюань --- это искусство 
с точки зрения боя, выведения противника, а точнее 
противников, которых должно быть много, то есть 
больше,  чем вас.  То есть их много, а нас мало. Это вот 
принцип тайцзицюань.  И задача тайцзицюань --- 
выведение противников из...\ как-бы лишение противников 
возможности продолжать бой через травмы, через 
увечья, через смерть. То есть... Вот идеальный бой тайцзицюань
--- это когда в начале боя, ну условно,  один, их 
десять, через пять минут или через три минуты вы один 
стоите на ногах, а десять лежат. Возможно, из этих 
десяти уже один не дышит. Остальные со сломанными 
руками, ногами, кого-то позвоночник поврежденный.  Ну, 
как бы вы защитили того, кого вы охраняли, а все 
остальное касается... Ещё раз, это если буквально 
следовать тем принципам, которые заложены в боевое 
искусство тайцзицюань. Вот наш учитель Григорьев 
Анатолий Алексеевич, он буквально со слов китайских 
мастеров однажды сказал... Он проходил и проходит 
обучение у мастеров из Гонконга другой школы тайцзицюань.
Там школа Хунгар, школа Тигра. И он сказал, что 
мастера Хунгар про тайзишников говорят буквально 
следующее: что у нас в стиле Хунгар или Хунцзя или
Хунцюань, у нас в стиле тренировки жёсткие, но 
в бою мы действуем мягко. Тайцзишники наоборот, на 
тренировках мягкие, а в бою действуют жестко. Есть 
контекст применения Тайцзи очень такой... То есть 
когда вы его будете полностью, по полной программе 
применять в боевой ситуации, то ваши противники 
должны быть просто не в состоянии шевелить руками и 
ногами. Это не контекст единоборств, обращаю внимание. 
Это не имеет никакого отношения к спорту. Спорт --- это 
соревнование, когда люди остаются не то что в живых, а 
в общем-то не сильно повреждёнными.

Соответственно,  
это большая проблема, глобальная проблема, что тайцзицюань
нельзя применять полностью в том виде, в котором 
он разрабатывался. И мы вынуждены адаптировать 
тайцзицюань. Мне очень нравится Тайцзи, я им занимаюсь 
больше 30 лет. Я начал в 35 лет, в 90-м году я пришёл в Тайцзи, 
вот уже 35 лет я занимаюсь этим искусством. Ну, с 
небольшими перерывами. Мне очень нравится эта школа,  
метод. Но я не могу применить это в реальности. Я это 
прекрасно понимаю, потому что у меня нет, извините, 
лицензии на убийство, лицензии на то, чтобы увечить 
людей, даже если бы я этого хотел. Отдельный вопрос, 
что я этого в общем-то и не хочу. Ну вот неважно, хочу 
или не хочу, права у меня такого нет. И, как говорится,  
если я начну проверять на практике, как работает 
Тайцзи на улицах, ну я могу сесть надолго и зачем мне 
это надо? Зачем нужно такой серьезный вопрос, который 
мы постоянно себе задаём. Этот вопрос обсуждается в 
сообществе мастеров боевых искусств, мастеров Тайцзи,
уже не первые десяток лет. Сколько я этим занимаюсь, 
столько этот вопрос поднимается. Зачем мы изучаем 
это боевое искусство и как изучая это боевое 
искусство потом применять его на улице. На улице или в 
какой-то реальной ситуации. Отдельная тема --- это 
соревнования по Тайцзи-Цюань. У нас 
невозможен спортивный вариант тайцзицюань. Мы не 
говорим сейчас про спортивное ушу. Спортивное ушу --- 
это не боевое искусство, это гимнастика. Если мы 
говорим про боевую искусство, то какие возможности 
соревнований? Только показательные выступления.  
Потому что, опять-таки,  полный контакт, да даже пусть 
не в полный контакт, невозможно соревноваться в 
тайцзицюань, потому что, ещё раз, это все приемы, все 
реальные движения травматичные. И не просто
травматичные, а нацеленные на то, чтобы травмировать 
противника. Это серьезный вопрос,  серьезный вызов 
такой, с которым мы разбираемся в преподавании. Кто-то 
не разбирается, кто-то просто выкидывает боевое 
искусство из преподавания тайцзицюань, остаётся 
только духовная практика и оздоровление. То есть, как-бы,
выкидываем из трёх одно, остаётся два, ну и 
нормально. Этого достаточно. Мне этого недостаточно, 
я не хочу бросать боевое искусство. Поэтому вопрос, 
как это преподавать? Ну, мы к этому вопросу, 
естественно, ещё будем возвращаться. Пока мы только 
этот вопрос обозначаем, что он есть, и что мы будем о 
нём думать и помнить.

Значит, третий, наверное, 
такой момент. Я уже обозначил два момента. Первый 
это вопрос культурного перевода.
Второй вопрос --- это 
очень сильные различия условий, в которых создавался и 
развивался тайцзицюань и современных условий.  Третий 
вопрос --- это вопрос, собственно, системы 
преподавания именно как методики. Методика 
преподавания тайцзицюань опять-таки очень сильно была 
связана с контекстом, в котором это всё создавалось. 
То есть вот если вы посмотрите китайские фильмы про 
историю тайцзицюань, там видно как обстояло 
дело, как люди изучали Тайцзи. Есть деревня Чэнь Цзя Гоу. 
На деревню нападают какие-нибудь то ли бандиты, то ли 
какие-нибудь странные люди, представители охраны
какого-нибудь князя, который непонятно имеет права на 
эту деревню, не имеет, неважно. Или какие-нибудь там 
манчжурские мажоры просто прискакали и начали просто 
всех бить, грабить, там ещё что-нибудь. Ну то есть 
жизнь была весёлая. И вот ты представитель клана Чэнь,  
дают тебе в руки саблю и говорят: ``Защищай родную деревню.''
И 
ты бежишь, размахиваешь этой саблей, защищаешь родную 
деревню.  Значит, если ты выжил после этого боя, ну 
там, их там 30 и нас 20, да, например.  Ну, за счёт того, что 
у нас есть несколько мастеров, остальные ученики, вот 
мы выжили. Но не все выжили. То есть ты выжил в первом 
бою. И вот если ты выжил в первом бою, то тебе начинают 
показывать приёмы. Тебе начинают показывать, как 
саблю держать. Потому что все боевые искусства... Это 
достоверно известно и доказано научными 
исследованиями. Все боевые искусства изначально были 
оружными стилями. То есть борьба очень сильно позднее 
добавлялась. Борьба, кулачный бой. Это как в 
современном спецназе есть такая шутка. Рукопашный бой 
используется спецназовцем в ситуации, когда он потерял 
автомат, пистолет, штык-нож, сапёрную лопатку, гранаты 
и встретил второго такого долбоёба, прошу прощения,  
который тоже всё это потерял. Вот тогда используется 
рукопашный бой. Но в этом смысле ничего не изменилось. 
Как и 200--300 лет назад, чтобы применять боевое искусство 
без оружия, нужно сначала оружие потерять, как-то 
оказаться в ситуации, что у вас нет оружия, а у 
противников оно есть. С чего бы это было --- непонятно.  
То соответственно, вот вы выжили в первом бою, вам 
показывают как оружие в руках держать. Вы выжили во 
втором бою, вам показывают как стойку нужно править.  
Вы выжили в течение года, вы уже более-менее 
подготовленный боец.  И фактически сроки подготовки в 
такой ситуации, когда регулярно идут стычки, 
регулярно идут какие-то реальные проверки, проверки в 
условиях жизни и смерти... В этой ситуации мастера, 
хорошо подготовленного бойца, делали скорее всего за 
пару лет.  Возможно год. Делали минимально 
подготовленного бойца, чтобы он хотя бы мог выжить и 
тренироваться дальше.  Два-три года был такой цикл 
подготовки бойца уже самодостаточного. В принципе,  
это соответствует опять-таки современным нормам. То 
есть когда Чэнь Юя, когда он приезжал к нам в Россию, 
Анатолий Алексеевич, наш учитель лично у него 
занимался, брал уроки и спросил, каковы сроки 
подготовки бойца. Чэнь Юй сказал, что если ко мне 
человек придёт в личные ученики, будет жить со мной, 
там, будет тренироваться каждый день и так далее, то 
есть будет полностью тренироваться по моим условиям, 
по моей методике, то через два года он будет бойцом. 
Через два года. Не 10 лет как сейчас идут разговоры, не 
всю жизнь --- два года. Но это тренировки с утра до вечера.

Немножко мы пытаемся приблизиться 
вот к этим как бы стандартам тренировок в рамках наших 
например интенсивов. Мы выезжаем летом на Алтай, а 
зимой мы собираемся в Новосибирске, проводим интенсивы 
несколько дней. Ну опять, что такое интенсив? Это 5--7 
часов тренировок. Ну, может быть 8, если люди хорошо 
подготовлены. Может быть 8 часов тренировок в день, но 
не больше. Больше 8 часов тренировок в день даже 
подготовленный человек долго не выдержит. Поэтому это 
как бы вот...\ интенсивные тренировки, но с разумной 
интенсивностью, чтобы не умирать, а сохранять 
работоспособность. Опять-таки в любой момент может 
быть очередной бой, очередная стычка. Если вы 
тренировались так, что не можете шевельнуть ни рукой, 
ни ногой, то это плохие тренировки, вредные. Вредные 
для вашего здоровья и выживания. Их, конечно, никто не 
будет организовывать.
То есть вот в этих условиях 
преподавание тайцзицюань --- это быстрый процесс.

В 
наших условиях, когда мы тренируемся в идеальном 
случае, в идеальном --- два часа в день. То есть вот у 
меня был идеальный режим тренировок, когда я был 
молодой, я сам тренировался по два часа в день. Плюс я 
ходил к учителю на тренировки, плюс я...\ занимался, мы 
встречались в парах, у нас были спарринги, парные 
тренировки. Ну в общей сумме доходило в неделю где-то 
до 20--25 часов тренировок. То есть соответственно 
порядка 100, может быть 120 часов тренировок в месяц. Вот, 
это для меня это...\ было максимум. Больше я не тянул, 
просто потому что мне, например, нужно было сначала 
учиться, потом работать. И я думаю, вы все согласитесь 
с тем, что больше двух часов тренировок каждый день 
выделять, ну это прям, это испытание. Это тяжело.  Вот. А 
в реальности люди у нас тренируются так: три-четыре 
часа тренировок в группе в неделю, то есть это...\ 
два-три занятия, по полтора-два часа. И дома 
какая-нибудь разминка, полчаса-час в день --- это 
максимум. То есть в реальности количество 
тренировочных часов,  которые человек уделяет в разы 
меньше, чем то количество, которое было по стандартам школы 
тайцзицюань, да и собственно любой другой школы боевых 
искусств.

Соответственно, у нас возникает проблема как 
обучать человека, чтобы он без вот этого постоянного, 
как бы, без постоянной нагрузки, без постоянной 
проверки на прочность, проверки на мастерство, на 
выживание, чтобы он был в состоянии развиваться.  И вот 
когда мы читаем...\ учебники по тайцзицюань, особенно 
старые учебники, трактаты по тайцзицюань, нужно 
понимать, что эти трактаты и учебники были написаны 
вот как раз для вот этого человека 17--18--19 века, который 
тренировался по 8 часов в день. Если человек 
работал в поле там или где-то, то 4 часа в день это было 
уже много.
Но... Вот для такого интенсивно 
занимающегося человека были написаны эти трактаты. 

Если сейчас современный преподаватель Тайцзи эти 
трактаты даёт ученикам и говорит: ``Читайте. Вот это 
истина и познав эту истину вы всё поймёте.'' --- это чушь 
собачья. Понимаете? Вот люди, которые так организуют 
тренировочный процесс, честно говоря, у меня это 
вызывает...\ недоумение. ``Вы не понимаете'' ---  мне хочется 
спросить коллег, которые преподают, вот опираясь на 
эти старые трактаты --- ``вы не понимаете, что современный 
ученик никак ничему не научится, если он будет эти 
трактаты читать, пусть даже будут учителя ему что-то 
объяснять и так далее.'' То есть современная методика 
преподавания должна быть написана заново для 
человека, который занимается в современных условиях. 

То же самое, мы проходили обучение, я проходил, Сергей 
Титов, проходили обучение в закрытой доосской школе. 
Значит, при всём моём уважении к наставникам этой 
закрытой школы, я хочу сказать, что они преподают по 
старым достаточно консервативным методикам. Ещё раз, 
методикам, которые были созданы там 200--300 лет назад. Не 
сильно эти методики адаптированы к современной 
реальности. Почему? А потому что --- а зачем?  Это же 
классика, так учили мастера.  Так учили мастера, 
значит и мы так дальше будем учить. Но вот для 
современного человека это не работает. Современному 
человеку --- я профессиональный психолог и 
профессиональный педагог --- я могу сказать, что 
современному человеку, особенно русскому 
современному человеку, невозможно эффективно что-то 
преподавать, так же как современному китайцу, и тем 
более, как китайцу 18--19 века. Невозможно. Если мы хотим, 
чтобы человек чему-то научился, преподавать нужно 
по-другому.

И вот в том числе в рамках вот этой идеи, 
что преподавать нужно по-другому, мы развиваем нашу 
школу «Сибирская Школа тайцзицюань», читаем вот эти 
лекции, записываем всякие видео. То есть вот регулярно 
звучат такие комментарии. ``А вот зачем вы записываете 
эти видео? Потому что это же профанация, потому что это 
же попса. Там вы ещё там в тиктоке начните записывать.''
У меня если бы время было, я бы и в тиктоке записывал. 
Только времени не хватает. Что у меня, что у Леры.
Мы как бы достаточно много работаем, только нам 
тиктока ещё не хватало. Но вообще-то по-хорошему, чтобы люди 
обучались, чтобы люди понимали что такое Тайцзи и как 
оно работает, нужно в том числе и в тиктоке 
присутствовать. То есть современный человек живет по 
принципу «чего нет там в ютубе или в контакте --- того нет 
в реальности». Понимаете? То есть если человек чего-то 
не видит в соцсетях,  то этого просто не существует в 
жизни.

Если мы не будем в соцсетях писать и говорить, 
не только «мы» в смысле вот Андрей Ширай и Школа, 
Сибирская Школа тайцзицюань, а «мы» преподаватели 
Тайцзи, не будем писать в соцсетях про это, потому что 
«это недостойно», потому что «это профанация», то просто 
никто не будет знать про тайцзицюань. То есть Тайцзи 
Цюань превратится в какое-то сектанство или клубы для 
идиотов, которые занимаются чем-то непонятным, навроде 
мормонов в Америке.  То есть, есть мормоны в Америке, у 
них даже какие-то книги написаны,  официальная 
мормонская церковь существует.  Это сумасшедшие 
такие, легализованные американские сумасшедшие.  И 
быть в Америке мормоном --- это приговор. Тебя никуда, 
как говорится, в приличное общество, скорее всего, не 
пустят.  Вот. Мы же не хотим, чтобы люди, занимающиеся 
тайцзицюань и другими традиционными боевыми 
искусствами, были аналогом мормонов в Америке. Мы 
хотим, чтобы тайцзицюань было...\ по возможности 
открытым движением, по возможности как бы не 
милитаризованным,  да, то есть не для военных, не для 
спецназа, а для людей. Прежде всего для людей. Потому 
что у военных и у спецназа есть свои системы, они 
весьма эффективные,  и там как раз никаких проблем, как 
говорится, с лицензией на убийство нет, поэтому там и 
методики тренировки соответствующие.

То есть вот 
давайте подводить итог.  Мы сегодня обозначили три 
проблемы преподавания Тайцзи. Я хочу сразу, ну может 
быть, ответить на вопросы, которые могут появиться.  А 
имеет ли смысл тогда заниматься тайцзицюань, если всё
так сложно и всё так плохо? Лучше тогда заниматься, там, 
боксом, MMA или каким-нибудь русским стилем 
типа Кадочникова системой.  Ну, я на эти вопросы не 
могу ответить. Почему? Во-первых, мне нравится Тайцзи 
Цюань, и я преподаю его, потому что мне нравится 
самому. Я не буду преподавать бокс или MMA или систему 
Кадочникова, потому что мне это не интересно. Вот. Я 
много чего попробовал в своей жизни. И рукопашный бой 
попробовал,  и Айкидо. Всякое было. Но вот в итоге я 
остановился на Тайцзи, ну плюс ещё парочка школ, 
которые к Тайцзи близки.  И мы занимаемся этим, потому 
что это нам нравится. Вот теперь вопрос. Если нам это 
нравится, как сделать так, это было доступно и 
возможно  передать людям? Не замороченным на тайцзицюань,
не повёрнутым, как мы.  Инструктора тайцзицюань
--- это немножко фанатики. Но в этом смысле любой 
инструктор --- это немножко фанатик. Я думаю в 
фитнес-клубе хороший инструктор тоже должен быть 
фанатиком, иначе он плохой фитнес-инструктор.  Поэтому 
инструктора фанатики. Но вот задача, чтобы тайцзицюань
был не только для фанатиков, а для нормальных 
людей, которые зачем-то пришли в нашу Школу и хотят 
какие-то цели достичь с помощью тайцзицюань.

Вот о том, 
как эти цели достижимы с помощью тайцзицюань, мы будем 
говорить на дальнейших лекциях.  Вот. На этом я 
замолкаю, но вы можете поднять руку и задать вопросы.  
Либо если вопросы... Я сейчас зайду в чат, посмотрю. 
Если вопросы кто-то в чате написал, то я... Отвечу на 
вопросы, заданные в чате. По-моему, в чате вопросов нет. 
Вопрос. Ну вот.  Поднимаем руку.  Мы включим микрофон и 
соответственно можно будет вопрос.  Напоминаю, для 
статистики ждём от вас потом в чате, в комментариях,  
какое у вас образование, кем вы работаете, профессия. 
Просто чтобы понимать,  какой у нас социальный состав 
тех, кто интересуется Тайцзи. Я, в общем-то, примерно и 
так знаю,  но лишний раз убедиться не помешает. Нет 
вопросов. Всё понятно или ничего не понятно.  Одно из 
двух.  Лера, выручай. У тебя точно есть вопрос. Ну 
хорошо. Надеюсь,  что моя сегодняшняя болотовня была в 
чём-то полезной.

Ещё раз я хочу обозначить,  что 
лекции, которые я читаю, не являются академически 
оформленным материалом. У меня могут быть в лекциях 
небольшие ---  не думаю, что принципиальные, но небольшие --- 
фактические ошибки, я могу где-то там какие-то даты 
перепутать или имена или что-то ещё.  В дальнейшем на 
основе лекций мы планируем издать книгу о 
тайцзицюань, поэтому любые ошибки, которые вы 
услышите, обнаружите в лекциях, с большой 
благодарностью мы примем критику,  комментарии о 
обнаруженных вами ошибках. Пожалуйста, кто понимает, о 
чём я говорю, не стесняйтесь писать,  критиковать.  Я 
вполне понимаю, что... Я сам слышу, когда я рассказываю.
Устная речь, она как бы бывает иногда нелогичной.  
Когда мы потом переводим это в письменную речь, в 
текст, приходится очень сильно работать над текстом, 
потому что многие логические нестыковки становятся 
заметными.

\bye
